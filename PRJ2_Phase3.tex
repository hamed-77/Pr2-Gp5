\documentclass[]{article}
\usepackage{lmodern}
\usepackage{amssymb,amsmath}
\usepackage{ifxetex,ifluatex}
\usepackage{fixltx2e} % provides \textsubscript
\ifnum 0\ifxetex 1\fi\ifluatex 1\fi=0 % if pdftex
  \usepackage[T1]{fontenc}
  \usepackage[utf8]{inputenc}
\else % if luatex or xelatex
  \ifxetex
    \usepackage{mathspec}
  \else
    \usepackage{fontspec}
  \fi
  \defaultfontfeatures{Ligatures=TeX,Scale=MatchLowercase}
\fi
% use upquote if available, for straight quotes in verbatim environments
\IfFileExists{upquote.sty}{\usepackage{upquote}}{}
% use microtype if available
\IfFileExists{microtype.sty}{%
\usepackage{microtype}
\UseMicrotypeSet[protrusion]{basicmath} % disable protrusion for tt fonts
}{}
\usepackage[unicode=true]{hyperref}
\hypersetup{
            pdfborder={0 0 0},
            breaklinks=true}
\urlstyle{same}  % don't use monospace font for urls
\usepackage{graphicx,grffile}
\makeatletter
\def\maxwidth{\ifdim\Gin@nat@width>\linewidth\linewidth\else\Gin@nat@width\fi}
\def\maxheight{\ifdim\Gin@nat@height>\textheight\textheight\else\Gin@nat@height\fi}
\makeatother
% Scale images if necessary, so that they will not overflow the page
% margins by default, and it is still possible to overwrite the defaults
% using explicit options in \includegraphics[width, height, ...]{}
\setkeys{Gin}{width=\maxwidth,height=\maxheight,keepaspectratio}
\IfFileExists{parskip.sty}{%
\usepackage{parskip}
}{% else
\setlength{\parindent}{0pt}
\setlength{\parskip}{6pt plus 2pt minus 1pt}
}
\setlength{\emergencystretch}{3em}  % prevent overfull lines
\providecommand{\tightlist}{%
  \setlength{\itemsep}{0pt}\setlength{\parskip}{0pt}}
\setcounter{secnumdepth}{0}
% Redefines (sub)paragraphs to behave more like sections
\ifx\paragraph\undefined\else
\let\oldparagraph\paragraph
\renewcommand{\paragraph}[1]{\oldparagraph{#1}\mbox{}}
\fi
\ifx\subparagraph\undefined\else
\let\oldsubparagraph\subparagraph
\renewcommand{\subparagraph}[1]{\oldsubparagraph{#1}\mbox{}}
\fi

\date{}

\begin{document}

\includegraphics[width=2.66389in,height=2.57708in]{media/image1.png}

\section{مقدمه}\label{ux645ux642ux62fux645ux647}

در فاز سوم که قسمت نهایی از انجام پروژه دوم می‌باشد تلاش ما بر این است
تا با بررسی جوانب گوناگون کسب‌وکار مورد بررسی بتوانیم نمودار BPMN مناسب
و بهینه را طراحی و گزارش بکنیم. بنابراین تلاش ما بر این است تا با بررسی
سناریو‌های ممکن برای پلترم فروش لباس خود بتوانیم با بررسی ابعاد گوناگون
نمودار خود را با توجه به استاندارد BPMN 2.0 طراحی بکنیم.

همانند فازهای پیش، پیش از معرفی بخش‌های مختلف سند فاز سوم خود، در اولین
بخش معرفی کوتاهی از نحوه‌ی انجام این فاز از پروژه و تخصیص فعالیت بین
اعضای گروه خواهیم داشت و سپس با جزئیات بخش‌های مختلف سند خود را ارائه
خواهیم کرد.

\section{روش انجام
پروژه}\label{ux631ux648ux634-ux627ux646ux62cux627ux645-ux67eux631ux648ux698ux647}

در این فاز پروژه نیز تلاش ما بر این است تا با استفاده از روش و متدولوژی
چابک یا همان Agile فعالیت‌های مورد نیاز پروژه را انجام دهیم و کار را پیش
بریم. در راستای این امر از دو ابزار به طور گسترده استفاده کرده‌ایم تا
بتوانیم هماهنگی لازم بین اعضای گروه را به خوبی مدیریت کنیم. اولین مورد
بستر Github است که از آن جهت بارگذاری خروجی‌ها و یکپارچه‌سازی فعالیت‌های
استفاده کردیم و دومین ابزار نیز بٌرد کانبان است که از آن جهت برنامه‌ریزی
مناسب فعالیت‌ها و رصد آنها استفاده کرده‌ایم. برای مثال در تصویر زیر
می‌توان نمایی از خروجی کانبان در اولین ساعت‌های انجام پروژه را مشاهده
کرد:

\includegraphics[width=6.50000in,height=2.99236in]{media/image2.png}

تصویر 1 -- Taskboard ابتدای پروژه

همانطور که مشاهده می‌شود فعالیتی در ابتدای امر انجام نشده‌است و تمامی
تسک‌ها نیازمند اجرا هستند. در ادامه به معرفی تسک‌های این پروژه و نحوه‌ی
شسکت‌ آنها خواهیم پرداخت.

\subsection{تسک‌های
پروژه}\label{ux62aux633ux6a9ux647ux627ux6cc-ux67eux631ux648ux698ux647}

فعالیت‌های مورد نیاز برای انجام این پروژه بر اساس شاخص‌های زیر شکسته
شده‌اند:

\begin{itemize}
\item
  نمودارهای خواسته شده توسط پروژه
\item
  سناریوهای موجود برای هر نمودار
\end{itemize}

در نتیجه سعی ما بر این بوده که در ابتدا تمامی تسک‌های مرتبط به هر دسته
از نمودارها را جدا بکنیم و در نهایت فعالیت‌های هر دسته را با توجه به
سناریوهای موجود در بین آنها بشکانیم. با توجه به توضیحات ارائه شده
فعالیت‌های مورد نیاز برای انجام فاز اول پروژه به شرح زیر است (با توجه به
استفاده از بستر Github عنوان این فعالیت‌ها به زبان انگلیسی تدوین
شده‌است):

\begin{enumerate}
\def\labelenumi{\arabic{enumi})}
\item
  Learning the software
\item
  Analyzing the similar BPMN charts
\item
  Drawing the BPMN Order chart
\item
  Drawing the BPMN Return chart
\item
  Drawing the BPMN Registration chart
\item
  Drawing Burndown chart
\item
  Sprint 3 report
\item
  Final Report
\item
  Aggregation of UML diagrams
\end{enumerate}

نکته حائز اهمیت این است که هر دسته از نمودرها به دو نفر از اعضای گروه
تخصیص یافت و به همین دلیل حجم کاری به خوبی مابین اعضای گروه پخش شده‌است.
در بخش بعدی به بیان زمان‌بندی اسپرینت دوم پروژه و نکات مرتبط با آن
خواهیم پرداخت.

\subsection{زمان‌بندی اجرای
پروژه}\label{ux632ux645ux627ux646ux628ux646ux62fux6cc-ux627ux62cux631ux627ux6cc-ux67eux631ux648ux698ux647}

مدت زمان پیشبینی شده برای انجام این فاز از پروژه پروژه ده روز است که
مطابق با برنامه‌ریزی زمان این اسپرینت از شنبه مورخ 20 دی الی دوشنبه مورخ
28 دی در نظر گرفته شده‌است. بنابراین این بخش در 10 روز و در قالب تسک‌های
کوتاه‌مدت انجام خواهد شد.

\subsection{نمودرهای
BPMN}\label{ux646ux645ux648ux62fux631ux647ux627ux6cc-bpmn}

در این بخش به بررسی نمودارهای BPMN مربوط به فرایندهای پلتفرم فروش
اینترنتی لباس خواهیم پرداخت:

\subsubsection{توضیحات و تصویر نمودار Order Fulfillment
Process}\label{ux62aux648ux636ux6ccux62dux627ux62a-ux648-ux62aux635ux648ux6ccux631-ux646ux645ux648ux62fux627ux631-order-fulfillment-process}

این قسمت از دو pool که اجزای کلی سیستم هستند یعنی پلتفرم و کاربر(مشتری)
تشکیل شده‌است؛ بخش پلتفرم خود از سه lane فروشگاه، پیک و سیستم یا همان
سامانه تشکیل شده است. علت این انتخاب در این بخش و بخش‌های بعدی این است
که مشتری به صورت کاملاً مستقل عمل می‌کند و با کمی ساده‌سازی می‌توان
اینگونه فرض کرد که سایر اجزای سیستم به مشتری خدمت می‌دهند و به همین دلیل
آنها را نیز در یک pool جداگانه در نظر گرفته‌ایم. استقلال عملیات‌های هر
بخش و فرایندهای موجود ملاک اصلی جهت تعریف pool و lane می‌باشد.

فرایند با ایونت بازدید مشتری از سایت استارت می‌خورد و پس از آن در اکانت
خود لاگین می‌کند؛ پس از لاگین کاربر، طبق متن پروژه سیستم موظف است اسامی
فروشگاه‌های در دسترس را به همراه هزینه پیک هر کدام نمایش دهد. پس از این
کاربر باید فروشگاه دلخواه خود را انتخاب کند و سپس سیستم باید محصولات
موجود در آن فروشگاه را از طریق دیتابیس بخواند و به کاربر نمایش دهد. در
این بخش یک گیت‌وی در نظر گرفته شده‌است که می‌خواهد ببیند آیا محصول مورد
نظر کاربر موجود است یا خیر، که اگر موجود نباشد کاربر از سفارش خود پشیمان
می‌شود و به نحوی فرایند کنسل می‌شود؛ امام اگر موجود باشد، کاربر آن را به
درون سبد خرید خود اضافه می‌کند. به عبارتی در این بخش فرض شده‌است که اگر
محصول موجود نباشد فرایند خرید لباس کنسل می‌شود.

پس از اضافه شدن محصول به سبد خرید توسط کاربر، سیستم نیز این عمل را انجام
می‌دهد و محصول را به سبد کاربر اضافه می‌کند که توضیح داده شده منظور از
سبد، سبد خرید است.

حال مجددا طبق فرضیات متن پروژه، سیستم موظف است سبد خرید را بررسی کند تا
ببیند مشکلی در آن وجود نداشته باشد (مثلا عدم تطابق ساعت سفارش با ساعت
کاری فروشگاه و ...)، اگر مشکلی وجود داشته باشد کاربر به یک مرحله قبل از
انتخاب فروشگاه مورد نظر برمی‌گردد و مجددا باید فروشگاه مورد نظر خود را
انتخاب کند و ثبت سفارش را انجام دهد؛ و اگر هم مشکلی در سبد خرید موجود
نباشد، سیستم کاربر را به صفحه پرداخت منتقل می‌کند.

حال کاربر موظف است که پرداخت را انجام بدهد و مدتی را منتظر بماند تا
سفارش به دستش برسد که در اینجا ما از ایونت میانی انتظار استفاده
کرده‌ایم.

پس از اتمام پرداخت کاربر سیستم نیز وظیفه دارد تا عملیات پرداخت را بررسی
کند؛ اگر پرداخت ناموفق باشد طبق فرض پروژه سبد خرید را ذخیره می‌کند و سپس
عملیات ثبت سفارش را پایان می‌دهد (در اینجا ما از ایونت خطا استفاده
کرده‌ایم به علت اینکه سیستم به کاربر پیام خطا را نمایش می‌دهد که پرداخت
ناموفق بوده است و سفارش پایان می‌پذیرد). بنابراین در این بخش فرض بر این
است که در صورت خطا در پرداخت عملیات خرید به اتمام می‌رسد.

اما اگر پرداخت با موفقیت انجام شده باشد، سیستم دو کار را به صورت همزمان
انجام می‌دهد اول اینکه لیست سفارشات مشتری به صاحب فروشگاه ارسال می‌شود و
کارهای فروشگاه استارت می‌خورد (از ایونت ارسال پیام استفاده کرده‌ایم) تا
فروشگاه آماده‌سازی سفارش را انجام دهد و منتظر پیک بماند (ایونت انتظار) و
همچنین سیستم بر روی پیک‌های موتوری آنلاین و در دسترس جست‌وجویی انجام
می‌دهد تا نزدیکترین موتور برای ارسال سفارش انتخاب شود.

پس از جست‌وجو، سیستم درخواست حمل سفارش را برای پیک مطلوب ارسال می‌کند و
یک پیام درخواست برای پیک ارسال می‌شود و وظایف پیک استارت می‌خورد (ایونت
پیام) سپس پیک باید درخواست را بررسی کند و به آن پاسخ دهد، پیک می‌تواند
درخواست را قبول یا رد بکند؛ که اگر درخواست را رد بکند سیستم مجدد به قبل
از جستجو بر می‌گردد و مجددا جستجوی پیک را انجام می‌دهد، اما اگر پیک قبول
بکند، سیستم اطلاعات سفارش را برای پیک ارسال می‌کند.

سپس پیک به فروشگاه مورد نظر می‌رود و کالا را تحویل می‎گیرد (فروشگاه
سفارش را آماده کرده و تا الان منتظر آمدن پیک مانده بود)؛ سپس پیک کالا را
برای مشتری ارسال می‌کند و کالا را به مشتری که تا الان منتظر تحویل سفارش
مانده بود تحویل می‌دهد و فرایند ثبت سفارش پایان می‌یابد.

نمودار این بخش را می‌تواند در صفحه‌ی بعد این گزارش مشاهده کرد:

\includegraphics[width=10.44375in,height=4.35417in]{media/image3.png}

\subsubsection{توضیحات و تصویر نمودار Return of Goods
Process}\label{ux62aux648ux636ux6ccux62dux627ux62a-ux648-ux62aux635ux648ux6ccux631-ux646ux645ux648ux62fux627ux631-return-of-goods-process}

در این بخش نیز مانند بخش ثبت سفارش از دو pool که اجزای کلی سیستم هستند
یعنی پلتفرم و کاربر(مشتری) استفاده شده است؛ پلتفرم خود از سه lane
فروشگاه، پیک و سیستم یا همان سامانه تشکیل شده‌است.

ابتدا فرایند با یک ایونت بازدید وبسایت توسط مشتری استارت می‌خورد، سپس بر
صفحه مربوط به عودت یا مرجوعی کالا را باز می‌کند و لاگین می‌کند. ما فرض
گرفته‌ایم که تماس صوتی از طریق سایت است (چیزی مثل voip)، حال مشتری از
طریق سایت با پشتیبانی تماس می‌گیرد و علت مرجوعی و ... را توضیح می‌دهد.

سپس با توجه به اینکه این درخواست به سیستم می‌رسد، کار سیستم را با یک
ایونت پیام استارت زده‌ایم؛ پس از رسیدن درخواست به سیستم، سیستم موظف است
که درخواست را بررسی کند (بدیهی است سیستم در اینجا به معنای فردی است که
در پشت سیستم نشسته و تصمیم‌گیری می‌کند). اگر درخواست عودت یا مرجوعی کالا
توسط سیستم رد شود عملیات مرجوعی پایان می‌یابد؛ اما اگر تایید شود توسط
سیستم یک زمان برای تحویل کالا از مشتری تنظیم می‌شود. و همچنین سیستم به
صورت اتوماتیک وظیفه تحویل کالا را به یک پیک ابلاغ می‌کند و اطلاعات
مرجوعی را برای پیک ارسال می‌کند در نتیجه کار پیک با یک ایونت پیام استارت
می‌خورد و اطلاعات مرجوعی را دریافت می‌کند؛ حال پیک در زمان مقرر و ست شده
برای تحویل کالا به آدرس مشتری مراجعه می‌کند؛ مشتری که تا این لحظه منتظر
مانده‌است کالا را به پیک تحویل می‌دهد.

حال پیک که کالا را تحویل گرفته‌است به فروشگاه می‌رود و کالا را به
فروشگاه تحویل می‌دهد (یعنی فروشگاه کالا را تحویل می‌گیرد) و سپس سیستم
هزینه سفارش را به حساب مشتری واریز می‌کند و فرایند مرجوعی پایان می‌یابد.

نمودار این بخش را می‌توان در صفحه‌ی بعد این گزارش مشاهده کرد:

\subsubsection{}\label{section}

\subsubsection{\texorpdfstring{\protect\hypertarget{_Toc62248360}{}{\protect\hypertarget{_Toc62248870}{}{}}\protect\includegraphics[width=9.96597in,height=4.95833in]{media/image4.jpeg}}{}}\label{section-1}

\subsubsection{توضیحات و تصویر نمودار Customer Registration
Process}\label{ux62aux648ux636ux6ccux62dux627ux62a-ux648-ux62aux635ux648ux6ccux631-ux646ux645ux648ux62fux627ux631-customer-registration-process}

در این قسمت از دو pool که اجزای کلی سیستم هستند یعنی سیستم و
کاربر(مشتری) استفاده شده است.

فرایند ثبت نام کاربر یا مشتری ابتدا با یک ایونت بازدید سایت توسط مشتری
استارت می‌خورد؛ سپس کاربر بر روی گزینه ثبت‎نام کلیک می‌کند و سپس فرم
اطلاعات مربوطه را پر می‌کند.

حال سیستم اطلاعات را بررسی می‌کند تا اطلاعات مشکلی نداشته باشند، اگر
اطلاعات وارد شده صحیح نبودند سیستم با یک ایونت میانی نمایش خطای عدم صحت
اطلاعات کاربر را به یک مرحله قبل از پر کردن اطلاعات می‌برد و باید مجدد
اطلاعات خود را وارد کند و اگر اطلاعات صحیح بودند سیستم بررسی می‌کند تا
ببیند کاربر چه راه ارتباطی را برای خود انتخاب کرده (یعنی شماره همراه خود
را وارد کرده یا ایمیل) اگر ایمیل را انتخاب کرده بود کد تایید برایش ایمیل
و اگر شماره را وارد کرده بود برایش پیامک می‌شود.

سپس با یک ایونت میانی پیام کد تایید برای کاربر ارسال می‌شود، و کاربر
باید کد را وارد کند؛ پس از وارد کردن کد توسط کاربر، سیستم باید بررسی کند
تا ببیند کد وارد شده صحیح است یا نه.

اگر کد وارد شده صحیح نبود کاربر باید مجدد کد را وارد کند و اگر صحیح بود
اکانت کاربر با موفقیت توسط سیستم ثبت می‌شود و در دیتابیس ایجاد می‌شود و
فرایند ثبت نام کاربر پایان می‌یابد.

نمودار این بخش را می‌توان در صفحه‌ی بعد این گزارش مشاهده کرد

\subsubsection{}\label{section-2}

\subsubsection{\texorpdfstring{\protect\hypertarget{_Toc62248362}{}{\protect\hypertarget{_Toc62248872}{}{}}\protect\includegraphics[width=9.53333in,height=3.09375in]{media/image5.jpeg}}{}}\label{section-3}

\subsubsection{توضیحات و تصویر نمودار Market Registration
Process}\label{ux62aux648ux636ux6ccux62dux627ux62a-ux648-ux62aux635ux648ux6ccux631-ux646ux645ux648ux62fux627ux631-market-registration-process}

این بخش از دو pool که اجزای کلی سیستم هستند یعنی پلتفرم و فروشگاه تشکیل
شده‌است؛ پلتفرم خود از دو lane مسئول ثبت نام و سیستم یا همان سامانه
تشکیل شده‌است.

این فرایند که فرایند ثبت‌نام فروشگاه می‌باشد، در ابتدا با یک ایونت
بازدید از سایت شروع می‌شود و سپس مسئول فروشگاه نیازمندی‌های ثبت‌نام
حضوری را در سایت چک می‌کند و بعد یک زمان را برای مراجعه حضوری انتخاب
می‌کند (این قسمت را خودمان فرض گرفته‌ایم که افراد می‌توانند از بین چند
زمان، زمان مورد نظر خودشان را اتخاب کنند).

سپس مسئول فروشگاه در زمان مقرر به صورت حضوری مراجعه می‌کند و مدارک خود
را تحویل می‌دهد و طبق فرض ما مسئول ثبت‌نام، اطلاعات فروشگاه و مدارک را
در سیستم وارد می‌کند؛ سپس سیستم صحت اطلاعات را بررسی می‌کند و در صورت
عدم اعتبار، مسئول ثبت‌نام موظف است به فروشگاه اطلاع بدهد که همکاری بین
سایت و فروشگاه ناممکن است و فرایند ثبت‌نام کنسل می‌شود؛ اما اگر پس از
بررسی سیستم، اطلاعات مورد تایید باشند، مسئول ثبت‌نام قرار داد را
می‌نویسد و مسئول فروشگاه آن را امضا می‌کند.

پس از امضای قرار داد توسط مسئول فروشگاه، مسئول ثبت‌نام موظف است که یک
پنل برای فروشگاه در سیستم ایجاد کند و سپس مسئول فروشگاه از طریق پنلش
اطلاعات فروشگاه‌ و کالاهاش را در پنل وارد می‌کند و در دیتابیس فروشگاه‌ها
ذخیره می‌شود (بدیهی است که بعدا می‌تواند اطلاعات و محصولات را ویرایش
کند)؛ سپس فرایند ثبت‌نام فروشگاه پایان می‌یابد.

\subsubsection{توضیحات و تصویر نمودار Courier Registration
Process}\label{ux62aux648ux636ux6ccux62dux627ux62a-ux648-ux62aux635ux648ux6ccux631-ux646ux645ux648ux62fux627ux631-courier-registration-process}

این بخش از دو pool که اجزای کلی سیستم هستند یعنی پلتفرم و پیک موتوری
تشکیل شده‌است؛ پلتفرم خود از دو lane مسئول ثبت‌نام و سیستم یا همان
سامانه تشکیل شده‌است.

این فرایند که فرایند ثبت‎نام پیک موتوری می‌باشد، در ابتدا با یک ایونت
بازدید از سایت شروع می‌شود و سپس پیک نیازمندی‌های ثبت‌نام حضوری را در
سایت چک می‌کند و بعد یک زمان را برای مراجعه حضوری انتخاب می‌کند (این
قسمت را خودمان فرض گرفته‌ایم که افراد می‌توانند از بین چند زمان، زمان
مورد نظر خودشان را اتخاب کنند).

سپس پیک در زمان مقرر به صورت حضوری مراجعه می‌کند و مدارک خود را تحویل
می‌دهد و طبق فرض ما مسئول ثبت‌نام، اطلاعات پیک و مدارک را در سیستم وارد
می‌کند؛ سپس سیستم صحت اطلاعات را بررسی می‌کند و در صورت عدم اعتبار،
مسئول ثبت‌نام موظف است به پیک اطلاع بدهد که همکاری بین سایت و پیک ناممکن
است و فرایند ثبت نام کنسل می‌شود؛ اما اگر پس از بررسی سیستم، اطلاعات
مورد تایید باشند، مسئول ثبت نام قرار داد را می‌نویسد و پیک آن را امضا
می‌کند.

پس از امضای قرار داد توسط پیک، مسئول ثبت نام موظف است که یک حساب کاربری
برای پیک در سیستم ایجاد کند و در دیتابیس پیک‌های موتوری ذخیره می‌شود.

حال مسئول ثبت‌نام موظف است بررسی کند که پیک دارای دستگاه GPS می‌باشد یا
خیر، اگر خیر، مسئول ثبت‌نام یک دستگاه GPS به پیک می‌دهد و فرایند ثبت‌نام
پایان می‌یابد و فرایند ثبت نام پایان می‌یابد. بنابراین فرض دیگر در این
بخش این است که در صورت عدم وجود دستگاه GPS، کسب‌وکار ما آن را به پیک
تخصیص می‌دهد.

دو نمودار ذکر را می‌توان در صفحه‌ی بعدی این گزارش مشاهده کرد:

\includegraphics[width=8.99861in,height=2.59375in]{media/image6.png}\includegraphics[width=9.00000in,height=3.16667in]{media/image7.jpeg}

\subsubsection{فرضیات}\label{ux641ux631ux636ux6ccux627ux62a}

به صورت کلی فرضیات مهم فاز سوم به شرح زیر می‌باشد:

\begin{enumerate}
\def\labelenumi{\arabic{enumi})}
\item
  در فرایند مرجوعی کالا پس از بررسی صحیح بودن درخواست، پاسخ از طریق یک
  پیام به کاربر نمایش داده می‌شود و امکان بررسی مجدد نیست.
\item
  در ثبت‌نام پیک و فروشگاه‌ها سروری وجود دارد که از طریق API به
  سامانه‌های مورد نیاز وصل است و در صورت وجود مشکلی از ثبت‌نام پیشگیری
  می‌کند. برای مثال سرور اطلاعاتی مانند سوءپیشینه‌ی پیک‌ها را پیش از
  ثبت‌نام بررسی می‌کند.
\item
  کدی که برای مشتریان ارسال می‌شود توسط تولید‌کننده‌ی کد ایجاد می‌شود.
\item
  منظور از لیست سفارشات پیشین در نمودار مرجوعی کالا، لیست یا موجودیتی
  است که با استفاده از آن کارشناس بررسی می‌کند که آیا سفارش ذکر شده
  اعتبار دارد و یا خیر.
\item
  قرارداد باید توسط کارشناس، تنظیم و توسط کاربر(پیک/فروشگاه) تکمیل و
  تایید شود.
\item
  در مورد پیک و فروشگاه، کارشناس حاضر در دفتر حضوری اطلاعات را از کاربر
  دریافت کرده و وارد سیستم می‌کند.
\item
  پس از ورود مشخصات فروشگاه یا پیک، در صورتی که هر گونه ایراد یا
  ناهماهنگی در اطلاعات وارد شده وجود داشته باشد، سرور تصدیق یا سامانه‌ی
  استعلام آن را شناسایی می‌کند.
\item
  جهت ثبت‌نام فروشگاه‌ها و پیک، آنها ابتدا باید یک زمان مراجعه‌ی حضوری
  دریافت و آن را رزرو کنند. این اقدام برای هر دو دسته از طریق وب‌سایت
  انجام می‌شود.
\item
  پس از ثبت‌نام و ایجاد حساب کاربری برای فروشگاه، پیک و مشتریان آنها
  باید اطلاعات حساب خود را تکمیل کنند.
\item
  در صورتی که پیک‌ها دستگاه GPS نداشته باشند، کسب‌وکار ما به آنها یک
  وسیله‌ی مناسب برای رفع این نیاز تخصیص می‌دهد.
\end{enumerate}

\subsection{نمودار BurnDown
Chart}\label{ux646ux645ux648ux62fux627ux631-burndown-chart}

در پایان این فاز نیز نمودار burndown اجرای پروژه را نهایی می‌کنیم که
نسخه‌ی نهایی آن به شرح زیر می‌باشد:

نمودار 6

Taskboard نهایی نیز در پایان این فاز پروژه به شرح زیر است:

\includegraphics[width=6.50000in,height=3.02153in]{media/image8.png}

تصویر 2

\subsection{منابع و
مآخذ}\label{ux645ux646ux627ux628ux639-ux648-ux645ux622ux62eux630}

\begin{itemize}
\item
  کتاب MIS
\item
  اسلایدهای دستیار آموزشی
\end{itemize}

\begin{itemize}
\item
  \url{https://www.lucidchart.com/pages/bpmn}
\item
  \url{https://www.youtube.com/watch?v=BwkNceoybvA}
\item
  \url{https://www.bpmn.org/}
\item
  \url{https://www.visual-paradigm.com/guide/bpmn/what-is-bpmn/}
\end{itemize}

\end{document}
