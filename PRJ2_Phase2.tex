\documentclass[]{article}
\usepackage{lmodern}
\usepackage{amssymb,amsmath}
\usepackage{ifxetex,ifluatex}
\usepackage{fixltx2e} % provides \textsubscript
\ifnum 0\ifxetex 1\fi\ifluatex 1\fi=0 % if pdftex
  \usepackage[T1]{fontenc}
  \usepackage[utf8]{inputenc}
\else % if luatex or xelatex
  \ifxetex
    \usepackage{mathspec}
  \else
    \usepackage{fontspec}
  \fi
  \defaultfontfeatures{Ligatures=TeX,Scale=MatchLowercase}
\fi
% use upquote if available, for straight quotes in verbatim environments
\IfFileExists{upquote.sty}{\usepackage{upquote}}{}
% use microtype if available
\IfFileExists{microtype.sty}{%
\usepackage{microtype}
\UseMicrotypeSet[protrusion]{basicmath} % disable protrusion for tt fonts
}{}
\usepackage[unicode=true]{hyperref}
\hypersetup{
            pdfborder={0 0 0},
            breaklinks=true}
\urlstyle{same}  % don't use monospace font for urls
\usepackage{longtable,booktabs}
\usepackage{graphicx,grffile}
\makeatletter
\def\maxwidth{\ifdim\Gin@nat@width>\linewidth\linewidth\else\Gin@nat@width\fi}
\def\maxheight{\ifdim\Gin@nat@height>\textheight\textheight\else\Gin@nat@height\fi}
\makeatother
% Scale images if necessary, so that they will not overflow the page
% margins by default, and it is still possible to overwrite the defaults
% using explicit options in \includegraphics[width, height, ...]{}
\setkeys{Gin}{width=\maxwidth,height=\maxheight,keepaspectratio}
\IfFileExists{parskip.sty}{%
\usepackage{parskip}
}{% else
\setlength{\parindent}{0pt}
\setlength{\parskip}{6pt plus 2pt minus 1pt}
}
\setlength{\emergencystretch}{3em}  % prevent overfull lines
\providecommand{\tightlist}{%
  \setlength{\itemsep}{0pt}\setlength{\parskip}{0pt}}
\setcounter{secnumdepth}{0}
% Redefines (sub)paragraphs to behave more like sections
\ifx\paragraph\undefined\else
\let\oldparagraph\paragraph
\renewcommand{\paragraph}[1]{\oldparagraph{#1}\mbox{}}
\fi
\ifx\subparagraph\undefined\else
\let\oldsubparagraph\subparagraph
\renewcommand{\subparagraph}[1]{\oldsubparagraph{#1}\mbox{}}
\fi

\date{}

\begin{document}

\includegraphics[width=2.66389in,height=2.57708in]{media/image1.png}

\section{فاز دوم}\label{ux641ux627ux632-ux62fux648ux645}

\section{مقدمه}\label{ux645ux642ux62fux645ux647}

در فاز دوم که قسمت بعدی از انجام پروژه دوم می‌باشد تلاش ما بر این است تا
با بررسی جوانب گوناگون کسب‌وکار مورد بررسی بتوانیم نمودارهای UML مناسب و
بهینه را طراحی و گزارش بکنیم. بنابراین تلاش ما بر این است تا با بررسی
سناریو‌های ممکن برای پلترم فروش لباس خود بتوانیم با بررسی ابعاد گوناگون
نمودارهای پرکاربرد UseCase، Activity و Sequence را گزارش دهیم.

همانند فاز اول، پیش از معرفی بخش‌های مختلف سند فاز دوم خود، در اولین بخش
معرفی کوتاهی از نحوه‌ی انجام این فاز از پروژه و تخصیص فعالیت بین اعضای
گروه خواهیم داشت و سپس با جزئیات بخش‌های مختلف سند خود را ارائه خواهیم
کرد.

\section{روش انجام
پروژه}\label{ux631ux648ux634-ux627ux646ux62cux627ux645-ux67eux631ux648ux698ux647}

در این فاز پروژه نیز تلاش ما بر این است تا با استفاده از روش و متدولوژی
چابک یا همان Agile فعالیت‌های مورد نیاز پروژه را انجام دهیم و کار را پیش
بریم. در راستای این امر از دو ابزار به طور گسترده استفاده کرده‌ایم تا
بتوانیم هماهنگی لازم بین اعضای گروه را به خوبی مدیریت کنیم. اولین مورد
بستر Github است که از آن جهت بارگذاری خروجی‌ها و یکپارچه‌سازی فعالیت‌های
استفاده کردیم و دومین ابزار نیز بٌرد کانبان است که از آن جهت برنامه‌ریزی
مناسب فعالیت‌ها و رصد آنها استفاده کرده‌ایم. برای مثال در تصویر زیر
می‌توان نمایی از خروجی کانبان در اولین ساعت‌های انجام پروژه را مشاهده
کرد:

\includegraphics[width=6.50000in,height=3.04167in]{media/image2.png}

تصویر 1 - Taskboard ابتدای اجرای فاز دوم

همانطور که مشاهده می‌شود تقریباً فعالیتی در ابتدای امر انجام نشده‌است و
تمامی تسک‌ها نیازمند اجرا هستند. در ادامه به معرفی تسک‌های این پروژه و
نحوه‌ی شسکت‌ آنها خواهیم پرداخت.

\subsection{تسک‌های
پروژه}\label{ux62aux633ux6a9ux647ux627ux6cc-ux67eux631ux648ux698ux647}

فعالیت‌های مورد نیاز برای انجام این پروژه بر اساس شاخص‌های زیر شکسته
شده‌اند:

\begin{itemize}
\item
  نمودارهای خواسته شده توسط پروژه
\item
  سناریوهای موجود برای هر نمودار
\end{itemize}

در نتیجه سعی ما بر این بوده که در ابتدا تمامی تسک‌های مرتبط به هر دسته
از نمودارها را جدا بکنیم و در نهایت فعالیت‌های هر دسته را با توجه به
سناریوهای موجود در بین آنها بشکانیم. با توجه به توضیحات ارائه شده
فعالیت‌های مورد نیاز برای انجام فاز اول پروژه به شرح زیر است (با توجه به
استفاده از بستر Github عنوان این فعالیت‌ها به زبان انگلیسی تدوین
شده‌است):

\begin{enumerate}
\def\labelenumi{\arabic{enumi})}
\item
  Defining the actors of the system for UseCase diagram
\item
  Defining the requirements for each actor
\item
  Writing the table description of the UseCase diagram
\item
  Drawing the UseCase diagram
\item
  Writing the description of the Activity diagram for fist scenario
\item
  Drawing the Activity diagram for fist scenario
\item
  Writing the description of the Activity diagram for second scenario
\item
  Draw the Activity diagram for second scenario
\item
  Writing the description of the Activity diagram for third scenario
\item
  Draw the Activity diagram for third scenario
\item
  Writing the description of the Activity diagram for fist scenario
\item
  Draw the Activity diagram for fist scenario
\item
  Writing the description of the Activity diagram for second scenario
\item
  Draw the Activity diagram for second scenario
\item
  Writing the description of the Activity diagram for third scenario
\item
  Draw the Activity diagram for third scenario
\item
  Drawing sequence diagram for the second scenario
\item
  Drawing sequence diagram for the third scenario
\item
  Drawing a sequence diagram for the third scenario (Store)
\item
  Drawing a sequence diagram for the third scenario (Courier)
\item
  Reporting the Burn Down chart
\item
  Writing a Final Report
\end{enumerate}

نکته حائز اهمیت این است که هر دسته از نمودرها به دو نفر از اعضای گروه
تخصیص یافت و به همین دلیل حجم کاری به خوبی مابین اعضای گروه پخش شده‌است.
در بخش بعدی به بیان زمان‌بندی اسپرینت دوم پروژه و نکات مرتبط با آن
خواهیم پرداخت.

\subsection{زمان‌بندی اجرای
پروژه}\label{ux632ux645ux627ux646ux628ux646ux62fux6cc-ux627ux62cux631ux627ux6cc-ux67eux631ux648ux698ux647}

مدت زمان پیشبینی شده برای انجام این فاز از پروژه پروژه یک هفته است که
مطابق با برنامه‌ریزی زمان این اسپرینت از پنجشنبه مورخ 11 دی الی پنجشنبه
مورخ 18 دی در نظر گرفته شده‌است. بنابراین این بخش در 7 روز و در قالب
تسک‌های کوتاه‌مدت انجام خواهد شد.

\subsection{نمودار
UseCase}\label{ux646ux645ux648ux62fux627ux631-usecase}

\subsubsection{معرفی Actorها و نیازمندی
آنها}\label{ux645ux639ux631ux641ux6cc-actorux647ux627-ux648-ux646ux6ccux627ux632ux645ux646ux62fux6cc-ux622ux646ux647ux627}

در این بخش ابتدا باید Actorهای سیستم خود را معرفی بکنیم تا در ادامه
نیازمندی‌های آنها را با توجه به UseCase طراحی شده معرفی بکنیم. ما در این
بخش سیستم فروشگاه لباس آنلاین را مورد تحلیل قرار خواهیم داد.

همانند تمامی نمودارهای UseCase ابتدا باید به این نکته توجه کنیم که
Actorهای سیستم در دو دسته‌ی Primary و Secondary هستند. دسته‌ی اول
نقش‌هایی هستند که شروع فعالیت با سیستم را بر عهده دارند و دسته‌ی دوم
نیاز ایجاد شده را برطرف می‌کنند.

Actorهای دسته‌ی Primary سیستم مذکور به شرح زیر می‌باشند. نکته‌ی مهم در
دسته‌بندی زیر این است که سیستم فروشگاه اینترنتی لباس یک پلتفرم محسوب
می‌شود و بنابراین هر سه دسته‌ی زیر مشتری و شروع‌کننده فعالیت محسوب
می‌شوند، بنابراین با در نظر گرفتن کمی ساده‌سازی، بهتر است که تمامی آنها
را در این بخش قرار دهیم.

\begin{itemize}
\item
  مشتری (فردی که لباس را خریداری می‌کند)
\item
  فروشگاه لباس
\item
  پیک
\end{itemize}

جهت معرفی Actorهای دسته‌ی Secondary، میزان ساده‌سازی و جداسازی فعالیت‌ها
از اهمیت بسیار زیادی برخوردار است. با توجه به نکته‌ی ذکر شده لیست
Actorهای این بخش به صورت زیر خواهد بود:

\begin{itemize}
\item
  سامانه یا وب‌سرویس ارسال پیامک یا ایمیل (این بخش در واقعیت، عمدتاً
  برون‌سپاری می‌شود. به همین جهت به صورت جداگانه در نظر گرفته می‌شود)
\item
  بانک
\item
  سامانه (منظور سامانه‌ی هوشمندی است که وظیفه‌ی مدیریت پلتفرم و
  فعالیت‌های آن را دارد که به اختصار آن را سامانه تعریف می‌کنیم)
\item
  امور فروشندگان (با توجه به نوع فعالیت و تخصص مورد نیاز این بخش در
  سازمان‌ها جداگانه تعریف می‌شود و به همین جهت ما نیز آن را به صورت
  مستقل تعریف کرده‌ایم)
\item
  امور پیشتیبانی (این بخش که همان واحد پشتیبانی و CRM است باید جداگانه
  در نظر گرفته شود)
\item
  امور حمل‌ونقل (با توجه به نوع فعالیت و تخصص مورد نیاز این بخش در
  سازمان‌ها جداگانه تعریف می‌شود و به همین جهت ما نیز آن را به صورت
  مستقل تعریف کرده‌ایم)
\end{itemize}

حال که به صورت کامل Actorهای سیستم را تعریف کرده‌ایم باید نیازمندی‌های
مرتبط با هر کدام را تعریف بکنیم. در همین بخش UseCaseهایی که برای این
نیازمندی تعریف شده نیز معرفی خواهیم کرد.

\begin{itemize}
\item
  مشتری
\end{itemize}

مشتری ما، که مهمترین Actor ما نیز می‌باشد، در سیستم به صورت کلی
نیازمندی‌های زیر را دارد:

\begin{itemize}
\item
  او باید بتواند در پلتفرم ثبت‌نام بکند تا بتواند پس از تشکیل حساب
  کاربری، خرید خود را انجام دهد. بنابراین امکان ثبت‌نام، اولین نیازمندی
  او می‌باشد. در همین راستا ما اقدام به تعریف UseCase \emph{ثبت‌نام}
  کردیم که می‌تواند موفق یا ناموفق باشد که این موضوع را در بخش
  سناریو‌های جایگزین بررسی خواهیم کرد.
\item
  پس از ثبت‌نام، اگر مشتری بخواهد خرید خود را شروع بکند در اولین قدم
  باید بتواند که فروشگاه‌های نزدیک خود را به همراه کالاهای موجود در آن
  مشاهده بکند تا در نهایت فروشگاه مدنظر خود را انتخاب بکند تا بتواند سبد
  خرید خود را شکل دهد. به همین دلیل ما یک UseCase \emph{انتخاب فروشگاه و
  نمایش کالاهای موجود} را تعریف کرده‌ایم تا این نیازمندی را پوشش دهیم.
\item
  پس از انتخاب فروشگاه‌ها و علاقه‌مندی به لباس‌های خاص، مشتری نیاز دارد
  تا بتواند یک سبد خرید ایجاد بکند. به همین دلیل UseCase \emph{چیدن سبد
  خرید} را جهت پوشش این نیازمندی تعریف کردیم. نکته قابل توجه این است که
  این اتفاق باید پس از انتخاب فروشگاه و مشاهده لیست کالاهای موجود رخ
  دهد، بنابراین باید با UseCase انتخاب فروشگاه رابطه داشته باشد که این
  را در نمودار در نظر گرفته‌ایم.
\item
  ‌بعد از چیدن سبد خرید کاربر باید این امکان را داشته باشد تا بتواند
  خرید خود را نهایی کند تا در نهایت پرداخت را انجام دهد. بنابراین باید
  یک UseCase \emph{ثبت نهایی سبد خرید} در نظر گرفته شود تا در راستای آن
  کالاها بررسی شود و در صورت عدم وجود مشکل، تأیید سفارش داده شود.
  بنابراین بررسی کالاهای سفارش مشتری نیز دیگر مورد-کاربردی است که باید
  تعریف شود و سناریوهای جاگزین متناسب با آن تعریف شود که در ادامه به
  صورت کامل این موارد را بررسی خواهیم کرد.
\item
  در نهایت اگر خرید تأیید شود انتقال به درگاه بانکی و انجام عملیات واریز
  وجه باید انجام شود. بنابراین برای رفع این نیازمندی یک UseCase به نام
  \emph{انتقال به درگاه پرداخت و واریز وجه} تعریف شده‌است تا این بخش را
  با استفاده از آن پوشش دهیم. همانطور که می‌توان حدس زد، جهت اجرای این
  UseCase باید Actor بانک نیز درگیر شود تا عملیات پرداخت به درستی انجام
  شود. البته این بخش با توجه به احتمال شکست عملیات پرداخت، سناریو‌های
  جایگزینی نیز دارد که باید به آن نیز توجه شود.
\item
  در گام بعدی مشتری باید سفارش خود را از پیک دریافت کند. بنابراین جهت
  رسیدگی به این نیازمندی باید یک مورد-کاربرد مناسب تعریف بکنیم تا عملیات
  دریافت سفارش پوشش داده شود. \emph{تحویل سفارش به مشتری} UseCaseای است
  که در این بخش مورد استفاده قرار خواهد گرفت و رابطه‌ای مستقیمی نیز با
  Actor پیک خواهد داشت. با توجه به سناریو‌ی ذکر شده، کاربر باید بتواند
  پس از تحویل سفارش، نظر خود را نسبت به پیک و فروشگاه ثبت کند. بنابراین
  جهت پاسخ‌دهی به این نیاز باید UseCase \emph{اجرای نظرسنجی} را نیز
  تعریف بکنیم.
\item
  با توجه به سناریو‌ی تعریف شده، آخرین نیازمندی که یک کاربر در سیستم
  خواهد داشت امکان مرجوع کردن کالای خریداری شده‌است. کاربر باید درخواست
  خود را ثبت بکند بنابراین مورد-کاربرد \emph{تماس با پشتیبانی جهت مرجوعی
  کالا} تعریف شده‌است. در صورت تأیید این درخواست پیکی باید این کالا را
  از مشتری دریافت بکند بنابراین UseCase \emph{دریافت کالا از مشتری} جهت
  حل این موضوع را تعریف می‌کنیم. همانطور که می‌توان حدس زد این بخش نیز
  سناریو‌های جایگزینی خواهد داشت که در بخش مربوطه، آنها را بررسی خواهیم
  کرد.
\end{itemize}

\begin{itemize}
\item
  فروشگاه
\end{itemize}

در این بخش به بررسی نیازمندی‌های فروشگاه در پلتفرم خود خواهیم پرداخت:

\begin{itemize}
\item
  همانند مشتری، فروشگاه نیز باید بتواند ثبت‌نام خود را انجام دهد و
  اطلاعات خود وارد بکند. با توجه به این موضوع UseCase \emph{ثبت‌نام و
  ثبت اطلاعات فروشگاه} را جهت پاسخ به این نیاز تعریف کرده‌ایم.
\item
  در گام بعدی باید مسئول فروشگاه بتواند لیست کالاهای خود را در فروشگاه
  ثبت و ویرایش بکند. این نیازمندی از اهمیت بسیار زیادی برخوردار است و به
  جهت پاسخ‌دهی به آن UseCase \emph{اضافه و ویرایش لیست کالاها} را به
  نمودار خود اضافه می‌کنیم.
\item
  پس از ثبت سفارش توسط مشتری، فروشگاه باید از این موضوع باخبر شود و در
  راستای آن آماده‌سازی مورد نیاز را جهت تحویل سفارش انجام دهد. بنابراین
  دو UseCase \emph{اعلام سفارش به فروشگاه} و همچنین \emph{آماده‌سازی}
  برای این بخش تعریف می‌شوند. نکته‌ی مهم این است که این دو با یکدیگر
  رابطه دارند که آن را در نمودار مشخص کرده‌ایم. سمت دیگر UseCase اعلام
  سفارش نیز سیستم است که اجرای این مورد را بر عهده دارد.
\item
  بعد از آماده‌سازی فروشگاه باید سفارش را به پیک تخصیص‌یافته تحویل دهد
  تا ارسال سفارش به مشتری نهایی شود. به همین جهت یک UseCase بین دو Actor
  مشتری و پیک خود تعریف می‌کنیم که با استفاده از آن این نیازمندی را پاسخ
  دهیم. نام آن در نمودار ما \emph{دریافت و ارسال سفارش} می‌باشد. پس از
  دریافت سفارش توسط مشتری و ارسال نظر او نسبت به فروشگاه، مسئول فروشگاه
  می‌تواند به این نظر دسترسی داشته باشد و آن را مشاهده کند. این نیاز نیز
  از طریق مورد-کاربرد \emph{نظر به فروشگاه} پاسخ داده می‌شود.
\item
  آخرین نیازمندی یک فروشگاه در سیستم ما دریافت کالای مرجوعی از پیک است.
  این مورد نیز طریق UseCase \emph{تحویل کالا به فروشگاه} در نظر گفته
  شده‌است.
\end{itemize}

\begin{itemize}
\item
  پیک
\end{itemize}

پس از بررسی مشتری و فروشگاه حال نوبت Actor پیک است تا بررسی بخش Primary
خود را به اتمام برسانیم. نیازمندی‌های این بخش به شرح زیر می‌باشد:

\begin{itemize}
\item
  همانند دو Actor پیشین، پیک نیز نیازمند این مورد است تا بتواند ثبت‌نام
  خود را در پلتفرم ما انجام دهد. بنابراین UseCase \emph{ثبت‌نام و ثبت
  اطلاعات پیک} را برای این نقش باید در نظر گرفت.
\item
  مهمترین نیازمندی پیک در سامانه‌ی ما، مربوط به بخش ارسال سفارش‌های
  مشتریان است. در اولین گام مطابق با سناریو، پیک‌ها باید به سفارشات
  تخصیص داده شوند که در این عملیات سناریو‌ی جایگزین رد درخواست نیز باید
  در نظر گرفته شود. برای پاسخ به این بخش UseCase \emph{تخصیص نزدیک‌‌ترین
  پیک} را تعریف می‌کنیم.
\item
  پس از اجرای عملیات تخصیص، پیک باید سفارش را از فروشگاه دریافت بکند.
  این نیازمندی از طریق \emph{دریافت و ارسال سفارش}، که پیش‌تر نیز به آن
  اشاره کردیم، تأمین می‌شود. بلافاصله در گام بعدی، این سفارش باید به
  مشتری تحویل داده شود که این نیاز نیز از طریق مورد-کاربرد ت\emph{حویل
  سفارش به مشتری} در سیستم نمایش داده شده‌است. پس از این اتفاق، همانند
  فروشگاه‌ها، پیک‌ها نیز می‌توانند به نظراتی که کاربران در مورد آنها
  داده‌اند دسترسی داشته باشند و این مورد نیز به وسیله‌ی UseCase
  \emph{نظر به پیک} نمایش داده شده‌است.
\item
  آخرین نیازمندی یک پیک مربوط به بخش تحویل کالای مرجوعی به پیک است که
  دقیقاً برعکس سناریوی تحویل سفارش به مشتری است. برای این بخش ابتدا
  UseCase \emph{تخصیص پیک و دریافت کالا از مشتری} نیازمندی بخش اول این
  زنجیره را پاسخ می‌دهد و در راستای آن UseCase \emph{تحویل به فروشگاه}
  نیاز بخش دوم این زنجیره را پاسخ می‌دهد.
\end{itemize}

بنابراین با توجه به توضیحات ذکر شده، به صورت کامل نشان داده شد که سیستم
طراحی شده چگونه نیازهای Actorهای بخش Primary را پاسخ می‌دهد. حال به
بررسی بخش Secondary می‌پردازیم.

\begin{itemize}
\item
  سامانه یا وب‌سرویس ارسال پیامک یا ایمیل
\end{itemize}

همانطور که در ابتدا نیز ذکر کردیم، این نقش با توجه به اینکه توسط یک
سیستم یا فرد خارجی تأمین می‌شود، باید آن را به صورت مستقل مورد بررسی
قرار دهیم. نیازمندی این Actor بسیار مختصر و به شرح زیر است:

\begin{itemize}
\item
  این سامانه در هنگام ثبت‌نام مشتریان باید یک ایمیل یا پیامک جهت تأیید
  ساخت حساب کاربری ارسال بکند. بنابراین این نیازمندی را از طریق UseCase
  \emph{تأیید کردن} که با ثبت‌نام رابطه دارد نشان داده‌ایم.
\end{itemize}

\begin{itemize}
\item
  بانک
\end{itemize}

این Actor جهت انجام کارهای مربوط به پرداخت تعریف شده‌است:

\begin{itemize}
\item
  تنها نیازمندی این بخش مربوط به عملیات پرداخت سفارش است که باید توسط
  مشتریان انجام شود. بنابراین UseCase \emph{انتقال به درگاه پرداخت و
  واریز} وجه علاوه بر مشتری باید به بانک نیز وصل باشد نا نیازمندی ذکر
  شده را پاسخ دهد. نکته مهم در این بخش این است که از عملیات بازگشت وجه
  هنگام مرجوع کردن کالا جهت ساده‌سازی سیستم و پیشگیری از پیچیده شدن آن
  صرف نظر کرده‌ایم.
\end{itemize}

\begin{itemize}
\item
  سامانه
\end{itemize}

در ابتدا نیز ذکر کردیم که منظور از سامانه، سیستم هوشمندی است که وظایف
مهم پلتفرم را انجام می‌دهد و نیاز است که آن را در نظر بگیریم.
نیازمندی‌های این Actor به شرح زیر است:

\begin{itemize}
\item
  پس از نهایی کردن سبد خرید توسط مشتری، سامانه‌ی هوشمند ما باید این سبد
  را بررسی کند تا در صورت وجود مشکل از انتقال به مرحله‌ی پرداخت پیشگیری
  کند. جهت رفع این نیازمندی سامانه، UseCase \emph{بررسی کالاهای سفارش
  مشتری} را تعریف می‌کنیم که توسط سیستم باید انجام شود و از این طریق
  می‌توان تأیید سفارش را از سیستم دریافت کرد.
\item
  پس از انجام عملیات پرداخت، سیستم باید به صورت خودکار پیامی را جهت
  آماده‌سازی به فروشگاه ارسال کند. این نیاز را از طریق UseCase
  \emph{اعلام سفارش به فروشگاه} که پیش‌تر نیز به آن اشاره کرده بودیم
  پاسخ خواهیم داد.
\item
  از سامانه انتظار می‌رود تا به صورت خودکار فرایند تخصیص پیک برای
  سفارش‌ها را انجام دهد و این نیازمندی یکی از مهمترین موارد مرتبط با
  سامانه‌ی ما می‌باشد. جهت رسیدگی به این مورد از طریق سیستم، دو
  مورد-کاربرد \emph{تخصیص نزدیک‌ترین پیک} را تعریف کرده‌ایم که یکی جهت
  ارسال سفارشات و دیگری مربوط به مرجوع کردن کالاها می‌باشد.
\item
  علاوه بر موراد گفته شده، سامانه باید درخواست مرجوعی ثبت شده توسط
  کارشناس پشتیبانی را بررسی کند و در صورت تأیید آن را به کارشناس اعلام
  بکند. این نیازمندی نیز از طریق UseCase \emph{تأیید مرجوعی} در سیستم
  پاسخ داده می‌شود.
\end{itemize}

\begin{itemize}
\item
  امور فروشندگان و امور حمل‌ونقل
\end{itemize}

در این بخش به صورت همزمان دو Actorای را بررسی می‌کنیم که نیازمندی آنها
بسیار به یکدیگر شباهت دارد.

\begin{itemize}
\item
  هنگام ثبت‌نام یک پیک و فروشگاه در سامانه باید مسئولی وجود داشته باشد
  که با توجه به حضوری بودن فرایند ثبت‌نام، فرایند آنها را انجام دهد و
  اطلاعات مورد نیاز را دریافت و ثبت کند. در نتیجه کارشناسان بخش امور
  حمل‌ونقل و فروشندگان در سیستم باید این بخش را انجام دهند. در همین
  راستا این نیازمندی آنها تحت عنوان دو UseCase \emph{ثبت‌نام و ثبت
  اطلاعات فروشگاه} و \emph{ثبت‌نام و ثبت اطلاعات پیک} در سیستم پاسخ داده
  خواهد شد.
\end{itemize}

\begin{itemize}
\item
  امور پشتیبانی
\end{itemize}

مشتریان جهت مرجوع کردن کالا باید با کارشناس امور پشتیبانی تماس برقرار
کنند تا بتوانند فرایند مدنظر خودشان را انجام دهند. نیازمندی این Actor به
شرح زیر می‌باشد:

\begin{itemize}
\item
  جهت مرجوع کردن کالا، مشتری به کارشناس زنگ می‌زند تا او درخواست مرجوع
  کردن کالا را ثبت بکند. این نیازمندی از طریق مورد-کاربرد \emph{تماس با
  پشتیبانی جهت مرجوع کردن کالا} اجرایی می‌شود.
\end{itemize}

\subsubsection{جدول
توضیحات}\label{ux62cux62fux648ux644-ux62aux648ux636ux6ccux62dux627ux62a}

جدول توضیحات مربوط به نمودار UseCase نیز به شرح زیر است:

\begin{longtable}[]{@{}llll@{}}
\toprule
اکتورها & گام‌ها و سناریوهای جایگزین & توضیحات & اسم مورد
کاربرد\tabularnewline
\midrule
\endhead
مشتری & & در این بخش کاربر اطاعات هویتی خود را پر می‌کند و گزینه ثبت
اطلاعات را می‌زند. & ثبت نام\tabularnewline
سامانه یا وب سروی ارسال پیامک یا ایمیل & نمایش ارور عدم تایید اطلاعات &
با توجه به اینکه کاربر ایمیل یا شماره همراه خود را وارد کرده‌است برای او
از طریق وب سرویس یک ایمل و یا پیامک کد تایید ارسال می‌شود و کاربر باید
آن را وارد کند که در صورت غلط وارد شدن، \textbf{ارور عدم تایید اطلاعات}
برای کاربر نمایش داده می‌شود و به کاربر اجازه داده می‌شود دوباره تلاش
کند. & تایید کردن\tabularnewline
مشتری & & سامانه به صورت پیش‌فرض به کاربران یک صفحه نمایش می‌دهد که در
آن لیست فروشگاه‌ها و قیمت ارسال پیک آورده شده که کاربر فروشگاه مد نظر
خود را انتخاب می‌کند و کالاهای موجود را مشاهده می‌کند. & انتخاب فروشگاه
نمایش کالاهای موجود\tabularnewline
مشتری & & سپس کاربر کالاهای مد نظر خود و تعدادشان را مشخص می‌کند و در
سبد خریدش قرار می‌گیرند. & چیدن سبد خرید\tabularnewline
مشتری & & کاربر پس از چینش سبد خود روی گزینه ثبت نهایی کلیک می‌کند تا
سفارش توسط سیستم بررسی شود و در صورت نبود مشکل کاربر به صفحه پرداخت
هدایت شود. & ثبت نهایی سبد خرید\tabularnewline
سامانه & & در این قسمت سامانه لیست را بررسی می‌کند؛ از این جهت که تمامی
کالاهای درخواستی در فروشگاه مدنظر موجود باشد یا زمان سفارش در محدوده
زمان کاری فروشگاه قرار داشته باشد. & بررسی کالاهای

سفارش مشتری\tabularnewline
سیستم & رد سفارش & در این قسمت سیستم پس از بررسی کالاهای سفارش مشتری،
سفارش را تایید می‌کند هم در صورتی که بررسی مشکلی را مشاهده کند با
\textbf{رد سفارش} روبرو خواهیم شد. و کاربر باید سفارش خود را عوض کند. &
تایید سفارش\tabularnewline
مشتری

بانک & پرداخت ناموفق & پس از تایید سفارش، مشتری به در گاه پرداخت بانک
منتقل می‌شود و وجه را واریز می‌کند؛ در صورتی که این کار به درستی انجام
نشود \textbf{پرداخت ناموفق} می‌شود و عملیات خرید متوقف می‌شود ولی لیست
خرید کاربر نگه داشته می‌شود. & انتقال به درگاه

پرداخت و واریز وجه\tabularnewline
سامانه

فروشگاه لباس & & در صورتی که پرداخت موفق باشد سامانه لیست سفارش را به
فروشگاه اعلام می‌کند. & اعلام سفارش به

به فروشگاه\tabularnewline
فروشگاه لباس & & پس از اعلام سفارش به فروشگاه، فروشگاه سفارشات را آماده
سازی و بسته بندی می‌کند. & آماده سازی

سفارش\tabularnewline
سامانه

پیک موتوری & عدم تایید درخواست & پس از پرداخت موفق توسط مشتری سامانه به
صورت هم‌زمان بااعلام سفارش به فروشگاه، سامانه بر روی پیک‌های موتوری
آنلاین و در دسترس جست‌وجویی انجام می‌دهد تا نزدیک‌ترین موتور برای ارسال
سفارش انتخاب شود. در این قسمت یک درخواست برای پیک ارسال می‌شود و آدرس
فروشگاه نمایش داده می‌شود؛ پیک می‌تواند درخواست را قبول کند یا ممکن است
بر روی گزینه \textbf{عدم تایید درخواست} کلیک کند، در این صورت جستجویی
دیگر توسط سامانه انجام می‌شود. & تخصیص نزدیکترین پیک\tabularnewline
پیک موتوری

فروشگاه لباس & & درصورتی که پیک قبول کند اطلاعات سفارش برای او ارسال
می‌شود تا برود و کالا را تحویل بگیرد. & دریافت و ارسال

سفارش\tabularnewline
پیک موتوری

مشتری & & بعد از تحویل کالا از فروشگاه توسط پیک، پیک باید برود و کالا را
به مشتری تحویل دهد. & تحویل سفارش به

مشتری\tabularnewline
مشتری

امور پشتیبانی & & پس از تحویل کالا توسط مشتری، می‌تواند در صورت نارضایتی
یا ایراد کالا و ... کالا را مرجوع کند. به همین جهت در این قسمت مشتری با
تیم پشتیبانی تماس می‌گیرد و دلایل خود را جهت مرجوعی توضیح می‌دهد. & تماس
با پشتیبانی

جهت مرجوعی کالا\tabularnewline
امور پشتیبانی & عدم تایید مرجوعی & در صورتی که دلایل مطرح شده توسط مشتری
برای تیم پشتیبانی قابل قبول باشند؛ مرجوعی تایید می‌شود و درصورتی که
دلایل کافی نباشند مشتری با \textbf{عدم تایید مرجوعی} روبرو می‌شود و به
صورتی پیامکی بهش اطلاع داده می‌شود. & تایید مرجوعی\tabularnewline
امور پشتیبانی

پیک موتوری

مشتری & & پس از تایید مرجوعی توسط تیم پشتیبانی، امور پشتیبانی یک پیک را
مسئول می‌کند تا به منزل مشتری رفته و کالا را تحویل بگیرد. & تخصیص پیک و

دریافت کالا از مشتری\tabularnewline
پیک موتوری

فروشگاه لباس & & پس از تحویل کالای مشتری توسط پیک، پیک موتوری کالا را به
فروشگاه لباس مربوطه تحویل می‌دهد و فرایند مرجوعی پایان می‌پذیرد. & تحویل
کالا به فروشگاه\tabularnewline
امور فروشندگان

فروشگاه لباس & & فروشگاه‌ها برای ثبت نام و ثبت اطلاعات شرکت به صورت
حضوری به امور فروشندگان مراجعه می‌کنند و ثبت نام می‌شوند تا پنل در
اختیارشان قرار بگیرد. & ثبت نام و ثبت

اطلاعات فروشگاه\tabularnewline
فروشگاه لباس & & فروشگاه‌ها پس از ثبت نام و فعال شدن پنلشان می‌توانند
اقداماتی همچون اضافه و ویرایش لیست کالاها و ... را انجام دهند. & اضافه و
ویرایش

لیست کالاها\tabularnewline
امور حمل و نقل

پیک موتوری & & پیک‌های موتوی برای ثبت نام و ثبت اطلاعات وسیله نقلیه و
... به صورت حضوری به امور حمل و نقل مراجعه می‌کنند و ثبت نام می‌شوند و
سپس امکان فعالیت در سامانه برایشان فراهم می‌شود. & ثبت نام و ثبت

اطلاعات پیک\tabularnewline
مشتری & & پس از خرید هر مشتری و تحویل کالا، برای مشتری یک نظر سنجی ارسال
می‌شود تا به آن پاسخ بدهد. & اجرای نظر سنجی\tabularnewline
مشتری

فروشگاه لباس & & در یک بخش از نظر سنجی مشتری به ارزیابی فروشگاه لباس
وکالای خود می‌پردازد و نظر برای فروشگاه نیز ارسال می‌شود. & نظر به
فروشگاه لباس\tabularnewline
مشتری

پیک موتوری & & در بخش دیگر از نظر سنجی مشتری به ارزیابی پیک موتوری
می‌پردازد و نظر برای پیک نیز ارسال می‌شود. & نظر به پیک\tabularnewline
\bottomrule
\end{longtable}

جدول 1- جدول توضیحات نمودار UseCase

\subsubsection{نمودار طراحی
شده}\label{ux646ux645ux648ux62fux627ux631-ux637ux631ux627ux62dux6cc-ux634ux62fux647}

با توجه به توضیحات ارائه شده در دو بخش قبلی، نمودار UseCase پلتفرم فروش
آنلاین لباس به شرح زیر خواهد بود:

\includegraphics[width=6.50000in,height=8.45483in]{media/image3.jpeg}

نمودار 1 -- UseCase Diagram

\subsection{نمودارهای
Activity}\label{ux646ux645ux648ux62fux627ux631ux647ux627ux6cc-activity}

در این بخش با توجه به توضیحات ارائه شده در فایل پروژه، به معرفی و گزارش
5 نمودار توالی خواهیم پرداخت:

\begin{enumerate}
\def\labelenumi{\arabic{enumi})}
\item
  فرایند خرید لباس
\end{enumerate}

\includegraphics[width=6.81709in,height=7.69792in]{media/image4.jpeg}

نمودار 2

\begin{enumerate}
\def\labelenumi{\arabic{enumi})}
\item
  فرایند مرجوعی کالا و بازگشت به فروشگاه لباس مبدا
\end{enumerate}

\includegraphics[width=6.50000in,height=7.55507in]{media/image5.png}

نمودار 3

\begin{enumerate}
\def\labelenumi{\arabic{enumi})}
\item
  فرایند ثبت‌نام کاربران مختلف در سامانه

  \begin{itemize}
  \item
    مشتریان
  \end{itemize}
\end{enumerate}

\includegraphics[width=5.18750in,height=7.92452in]{media/image6.png}

نمودار 4

\begin{itemize}
\item
  فروشگاه
\end{itemize}

\includegraphics[width=5.78624in,height=7.85625in]{media/image7.jpeg}

نمودار 5

\begin{itemize}
\item
  پیک
\end{itemize}

\includegraphics[width=5.67928in,height=7.71389in]{media/image8.jpeg}

نمودار 6

\subsubsection{فرضیات}\label{ux641ux631ux636ux6ccux627ux62a}

\begin{enumerate}
\def\labelenumi{\arabic{enumi})}
\item
  مشتری برای خرید ابتدا وارد حساب کاربری خود می‌شود. سپس اطلاعات ورود آن
  بررسی و در صورت صحت یا عدم صحت اقدامات پس از آن صورت می‌گیرد.
\item
  قیمت پیک هر فروشگاه هنگام نمایش فروشگاه‌ها به کاربر نمایش داده می‌شود.
\item
  منظور از زمان سفارش در بخش خرید، همان زمان موجهی است که در متن پروژه
  به آن اشاره شده است.
\item
  در صورتی که سبد خرید یا زمان سفارش مشکلی داشت کاربر دوباره به چیدن سبد
  خرید و تعیین زمان سفارش باز می‌گردد.
\item
  بررسی صحت تراکنش توسط بانک انجام می‌شود.
\item
  جست و جوی پیک‌ها بر اساس معیارهای ذکرشده در صورت پروژه، مانند نزدیکی
  مکانی، صورت می‌پذیرد.
\item
  در صورتی که پیک قبول نکند جست و جو ادامه پیدا می‌کند.
\item
  مشتری برای ارجاع کالا به بخش مربوطه در سایت مراجعه می‌کند. سپس وارد
  حساب کاربری خود می‌شود تا درخواست خود را ثبت کند.
\item
  تیم پشتیبانی مسئولیت رسیدگی به ارجاع کالاها را به عهده دارد.
\item
  در مرحله‌ی مرجوعی کردن کالا، پیدا کردن پیک اندکی با مرحله‌ی سفارش کالا
  تفاوت دارد.
\item
  هزینه‌ی کالا در انتهای ارجاع کالا به مشتری عودت داده می‌شود.
\end{enumerate}

\begin{enumerate}
\def\labelenumi{\arabic{enumi})}
\item
  تمامی مشتریان، فروشگاه‌ها و پیک‌ها برای ثبت نام ابتدا به سایت مجموعه
  ما مراجعه می‌کنند. جهت ثبت‌نام فروشگاه‌ها و پیک، آنها ابتدا باید یک
  زمان مراجعه‌ی حضوری دریافت و آن را رزرو کنند. این اقدام برای هر دو
  دسته از طریق وب‌سایت انجام می‌شود.
\end{enumerate}

\begin{enumerate}
\def\labelenumi{\arabic{enumi})}
\item
  صحت‌سنجی اطلاعات و ارسال پیامک از طریق سیستم وب‌سایت انجام می‌شود.
\item
  در مورد پیک و فروشگاه، کارشناس حاضر در دفتر حضوری اطلاعات را از کاربر
  دریافت کرده و وارد سیستم می‌کند.
\item
  در صورتی که پیک‌ها دستگاه GPS نداشته باشند، کسب‌وکار ما به آنها یک
  وسیله‌ی مناسب برای رفع این نیاز تخصیص می‌دهد.
\item
  قرارداد باید توسط کارشناس، تنظیم و توسط کاربر(پیک/فروشگاه) تکمیل و
  تایید شود.
\end{enumerate}

\subsection{نمودارهای توالی
(Sequence)}\label{ux646ux645ux648ux62fux627ux631ux647ux627ux6cc-ux62aux648ux627ux644ux6cc-sequence}

در این بخش با توجه به توضیحات ارائه شده در فایل پروژه، به معرفی و گزارش
5 نمودار توالی خواهیم پرداخت:

\begin{enumerate}
\def\labelenumi{\arabic{enumi})}
\item
  فرایند خرید لباس
\end{enumerate}

\includegraphics[width=6.74994in,height=6.80208in]{media/image9.jpeg}

نمودار 7

\begin{enumerate}
\def\labelenumi{\arabic{enumi})}
\item
  فرایند مرجوعی کالا و بازگشت به فروشگاه لباس مبدا
\end{enumerate}

\includegraphics[width=6.50000in,height=6.31378in]{media/image10.jpeg}

نمودار 8

\begin{enumerate}
\def\labelenumi{\arabic{enumi})}
\item
  فرایند ثبت‌نام کاربران مختلف در سامانه

  \begin{itemize}
  \item
    مشتریان
  \end{itemize}
\end{enumerate}

\includegraphics[width=6.24375in,height=7.24348in]{media/image11.jpeg}

نمودار 9

\begin{itemize}
\item
  فروشگاه‌ها
\end{itemize}

\includegraphics[width=6.50000in,height=5.15120in]{media/image12.jpeg}

نمودار 10

\begin{itemize}
\item
  پیک
\end{itemize}

\includegraphics[width=6.50000in,height=5.95604in]{media/image13.jpeg}

نمودار 11

\subsubsection{فرضیات}\label{ux641ux631ux636ux6ccux627ux62a-1}

\begin{enumerate}
\def\labelenumi{\arabic{enumi})}
\item
  در فرایند مرجوعی کالا پس از بررسی صحیح بودن درخواست، پاسخ از طریق یک
  پیام به کاربر نمایش داده می‌شود و امکان بررسی مجدد نیست.
\item
  در ثبت‌نام پیک و فروشگاه‌ها سروری وجود دارد که از طریق API به
  سامانه‌های مورد نیاز وصل است و در صورت وجود مشکلی از ثبت‌نام پیشگیری
  می‌کند. برای مثال سرور اطلاعاتی مانند سوءپیشینه‌ی پیک‌ها را پیش از
  ثبت‌نام بررسی می‌کند.
\item
  کدی که برای مشتریان ارسال می‌شود توسط تولید‌کننده‌ی کد ایجاد می‌شود.
\item
  منظور از لیست سفارشات پیشین در نمودار مرجوعی کالا، لیست یا موجودیتی
  است که با استفاده از آن کارشناس بررسی می‌کند که آیا سفارش ذکر شده
  اعتبار دارد و یا خیر.
\item
  قرارداد باید توسط کارشناس، تنظیم و توسط کاربر(پیک/فروشگاه) تکمیل و
  تایید شود.
\item
  در مورد پیک و فروشگاه، کارشناس حاضر در دفتر حضوری اطلاعات را از کاربر
  دریافت کرده و وارد سیستم می‌کند.
\item
  پس از ورود مشخصات فروشگاه یا پیک، در صورتی که هر گونه ایراد یا
  ناهماهنگی در اطلاعات وارد شده وجود داشته باشد، سرور تصدیق یا سامانه‌ی
  استعلام آن را شناسایی می‌کند.
\item
  جهت ثبت‌نام فروشگاه‌ها و پیک، آنها ابتدا باید یک زمان مراجعه‌ی حضوری
  دریافت و آن را رزرو کنند. این اقدام برای هر دو دسته از طریق وب‌سایت
  انجام می‌شود.
\item
  پس از ثبت‌نام و ایجاد حساب کاربری برای فروشگاه، پیک و مشتریان آنها
  باید اطلاعات حساب خود را تکمیل کنند.
\item
  در صورتی که پیک‌ها دستگاه GPS نداشته باشند، کسب‌وکار ما به آنها یک
  وسیله‌ی مناسب برای رفع این نیاز تخصیص می‌دهد.
\end{enumerate}

\subsection{نمودار BurnDown
Chart}\label{ux646ux645ux648ux62fux627ux631-burndown-chart}

در پایان این فاز نیز نمودار burndown اجرای پروژه را نهایی می‌کنیم که
نسخه‌ی نهایی آن به شرح زیر می‌باشد:

نمودار 12

Taskboard نهایی نیز در پایان این فاز پروژه به شرح زیر است:

\includegraphics[width=6.50000in,height=2.99375in]{media/image14.png}

تصویر 2

\subsection{منابع و
مآخذ}\label{ux645ux646ux627ux628ux639-ux648-ux645ux622ux62eux630}

\begin{itemize}
\item
  کتاب MIS
\item
  اسلایدهای دستیار آموزشی
\end{itemize}

\begin{itemize}
\item
  \url{https://www.justinmind.com/blog/user-story-examples/}
\item
  \url{https://www.visual-paradigm.com/guide/uml-unified-modeling-language/what-is-use-case-diagram/}
\item
  \url{https://www.youtube.com/watch?v=zid-MVo7M-E}
\item
  \url{https://www.youtube.com/watch?v=18_kVlQMavE}
\item
  \url{https://www.youtube.com/watch?v=pCK6prSq8aw}
\item
  \url{https://www.youtube.com/watch?v=_Mzi1rYtI5U}
\item
  \url{https://www.visual-paradigm.com/guide/uml-unified-modeling-language/what-is-activity-diagram/}
\end{itemize}

\end{document}
