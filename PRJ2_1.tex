\documentclass[]{article}
\usepackage{lmodern}
\usepackage{amssymb,amsmath}
\usepackage{ifxetex,ifluatex}
\usepackage{fixltx2e} % provides \textsubscript
\ifnum 0\ifxetex 1\fi\ifluatex 1\fi=0 % if pdftex
  \usepackage[T1]{fontenc}
  \usepackage[utf8]{inputenc}
\else % if luatex or xelatex
  \ifxetex
    \usepackage{mathspec}
  \else
    \usepackage{fontspec}
  \fi
  \defaultfontfeatures{Ligatures=TeX,Scale=MatchLowercase}
\fi
% use upquote if available, for straight quotes in verbatim environments
\IfFileExists{upquote.sty}{\usepackage{upquote}}{}
% use microtype if available
\IfFileExists{microtype.sty}{%
\usepackage{microtype}
\UseMicrotypeSet[protrusion]{basicmath} % disable protrusion for tt fonts
}{}
\usepackage[unicode=true]{hyperref}
\hypersetup{
            pdfborder={0 0 0},
            breaklinks=true}
\urlstyle{same}  % don't use monospace font for urls
\usepackage{longtable,booktabs}
\usepackage{graphicx,grffile}
\makeatletter
\def\maxwidth{\ifdim\Gin@nat@width>\linewidth\linewidth\else\Gin@nat@width\fi}
\def\maxheight{\ifdim\Gin@nat@height>\textheight\textheight\else\Gin@nat@height\fi}
\makeatother
% Scale images if necessary, so that they will not overflow the page
% margins by default, and it is still possible to overwrite the defaults
% using explicit options in \includegraphics[width, height, ...]{}
\setkeys{Gin}{width=\maxwidth,height=\maxheight,keepaspectratio}
\IfFileExists{parskip.sty}{%
\usepackage{parskip}
}{% else
\setlength{\parindent}{0pt}
\setlength{\parskip}{6pt plus 2pt minus 1pt}
}
\setlength{\emergencystretch}{3em}  % prevent overfull lines
\providecommand{\tightlist}{%
  \setlength{\itemsep}{0pt}\setlength{\parskip}{0pt}}
\setcounter{secnumdepth}{0}
% Redefines (sub)paragraphs to behave more like sections
\ifx\paragraph\undefined\else
\let\oldparagraph\paragraph
\renewcommand{\paragraph}[1]{\oldparagraph{#1}\mbox{}}
\fi
\ifx\subparagraph\undefined\else
\let\oldsubparagraph\subparagraph
\renewcommand{\subparagraph}[1]{\oldsubparagraph{#1}\mbox{}}
\fi

\date{}

\begin{document}

\includegraphics[width=2.66389in,height=2.57708in]{media/image1.png}

\section{مقدمه}\label{ux645ux642ux62fux645ux647}

در فاز اول که اولین قسمت از پروژه دوم می‌باشد تلاش ما بر این است تا با
بررسی جوانب گوناگون کسب‌وکار مورد بررسی بتوانیم یک سند envisioning مناسب
و بهینه را طراحی و گزارش بکنیم. در این پروژه، کسب‌وکار مورد بررسی ما یک
پلتفرم فروش لباس است که در آن مشتریان و خریداران لباس را به فروشندگان و
فروشگاه‌های لباس متصل می‌کند و از طریق این زنجیره به خلق ارزش خواهد
پرداخت.

پیش از معرفی بخش‌های مختلف سند ارزیابی خود، در اولین بخش معرفی کوتاهی از
نحوه‌ی انجام پروژه و تخصیص فعالیت بین اعضای گروه خواهیم داشت و سپس با
جزئیات بخش‌های مختلف سند ارزیابی و امکان‌سنجی خود را ارائه خواهیم کرد.

\section{روش انجام
پروژه}\label{ux631ux648ux634-ux627ux646ux62cux627ux645-ux67eux631ux648ux698ux647}

در این پروژه تلاش ما بر این است تا با استفاده از روش و متدولوژی چابک یا
همان Agile فعالیت‌های مورد نیاز پروژه را انجام دهیم و کار را پیش بریم.
در راستای این امر از دو ابزار به طور گسترده استفاده کرده‌ایم تا بتوانیم
هماهنگی لازم بین اعضای گروه را به خوبی مدیریت کنیم. اولین مورد بستر
Github است که از آن جهت بارگذاری خروجی‌ها و یکپارچه‌سازی فعالیت‌های
استفاده کردیم و دومین ابزار نیز بٌرد کانبان است که از آن جهت برنامه‌ریزی
مناسب فعالیت‌ها و رصد آنها استفاده کرده‌ایم. برای مثال در تصویر زیر
می‌توان نمایی از خروجی کانبان در اولین ساعت‌های انجام پروژه را مشاهده
کرد:

\includegraphics[width=6.49931in,height=3.01042in]{media/image2.png}

تصویر 1 -- tasknoard اولیه پروژه

همانطور که مشاهده می‌شود فعالیتی در ابتدای امر انجام نشده‌است و تمامی
تسک‌ها نیازمند اجرا هستند. در ادامه به معرفی تسک‌های این پروژه و نحوه‌ی
شسکت‌ آنها خواهیم پرداخت.

\subsection{تسک‌های
پروژه}\label{ux62aux633ux6a9ux647ux627ux6cc-ux67eux631ux648ux698ux647}

فعالیت‌های مورد نیاز برای انجام این پروژه بر اساس شاخص‌های زیر شکسته
شده‌اند:

\begin{itemize}
\item
  ویژگی مشتریان و بخش‌بندی آنها
\item
  توالی موجود بین فعالیت و قابلیت شکست آنها
\end{itemize}

در نتیجه سعی ما بر این بوده که در ابتدا تمامی تسک‌های مرتبط به هر دسته
از مشتریان (خریداران، فروشگاه‌ها و پیک‌ها) را جدا بکنیم و در نهایت
فعالیت‌های هر دسته را با توجه به امکان ایجاد توالی در بین آنها بشکانیم.
با توجه به توضیحات ارائه شده فعالیت‌های مورد نیاز برای انجام فاز اول
پروژه به شرح زیر است (با توجه به استفاده از بستر Github عنوان این
فعالیت‌ها به زبان انگلیسی تدوین شده‌است):

\begin{enumerate}
\def\labelenumi{\arabic{enumi})}
\item
  Defining the areas of stakeholder values' of users
\item
  Defining the areas of stakeholder values' of markets
\item
  Defining the areas of stakeholder values' of couriers
\item
  Defining the sales features
\item
  Segmenting the users' market and figuring out the most important
  attributes of them
\item
  Segmenting the markets' market and figuring out the most important
  attributes of them
\item
  Segmenting the couriers' market and figuring out the most important
  attributes of them
\item
  Releasing the vision statement
\item
  Writing the user story for the section of users with the approach of
  figuring out the epics
\item
  Writing the user story for the section of markets with the approach of
  figuring out the epics
\item
  Writing the user story for the section of couriers with the approach
  of figuring out the epics
\item
  Defining the functional requirements related to section of users
\item
  Defining the functional requirements related to section of markets
\item
  Defining the functional requirements related to section of couriers
\item
  Identifying the time constraints
\item
  Identifying the scope constraints
\item
  Identifying the budget constraints
\item
  Identifying the other constraints
\item
  Writing a Final Report
\item
  Reporting the Burn Down chart
\end{enumerate}

نکته حائز اهمیت این است که هر دسته از مشتریان به یکی از اعضای گروه تخصیص
یافت و به همین دلیل حجم کاری به خوبی مابین اعضای گروه پخش شده‌است. در
بخش بعدی به بیان زمان‌بندی اسپرینت اول پروژه و نکات مرتبط با آن خواهیم
پرداخت.

\subsection{زمان‌بندی اجرای
پروژه}\label{ux632ux645ux627ux646ux628ux646ux62fux6cc-ux627ux62cux631ux627ux6cc-ux67eux631ux648ux698ux647}

مدت زمان پیشبینی شده برای انجام این پروژه یک هفته است که مطابق با
برنامه‌ریزی اولیه قرار بود از دوشنبه مورخ 1 دی الی دوشنبه 8 دی به طول
بینجامد. اما با همانگی با جناب آقای دکتر حبیبی، با توجه به وجود امتحان
میان‌ترم درس، برنامه‌ریزی زمانی مجدد انجام شده و زمان این اسپرینت
استثنائاً از جمعه مورخ 5 دی الی سه‌شنبه مورخ 9 دی در نظر گرفته شده‌است.
بنابراین این بخش در 5 روز و در قالب تسک‌های کوتاه‌مدت انجام خواهد شد.

\section{Product Vision}\label{product-vision}

در این بخش اطلاعاتی که مرتبط با Product Vision کسب‌وکار ما است را ارائه
خواهیم کرد. در اولین بخش که شاید یکی از مهترین بخش‌های این سند باشد به
بخش‌بندی بازار و مشتریان خود می‌پردازیم:

\subsection{بخش‌بندی بازار و معرفی کاربران
هدف}\label{ux628ux62eux634ux628ux646ux62fux6cc-ux628ux627ux632ux627ux631-ux648-ux645ux639ux631ux641ux6cc-ux6a9ux627ux631ux628ux631ux627ux646-ux647ux62fux641}

\subsubsection{بخش مشتری
(Users)}\label{ux628ux62eux634-ux645ux634ux62aux631ux6cc-users}

برای بخش‌بندی مشتریان خود که به عنوان بخش مشتریان و خریداران پلترفم ما
محسوب می‌شوند، با توجه به اینکه محصول اصلی که آنها می‌توانند از فروشگاه
آنلاین ما خرید بکنند لباس است، از ویژگی‌ها و Attributeهای زیر جهت
بخش‌بندی بازار خود استفاده می‌کنیم:

\begin{longtable}[]{@{}ll@{}}
\toprule
ویژگی یا Attribute & بخش‌های بازار با توجه به ویژگی ذکر
شده\tabularnewline
\midrule
\endhead
سن & \begin{itemize}
\item
  جوان
\item
  میان‌سال
\item
  پیر
\end{itemize}\tabularnewline
سطح دارمد & \begin{itemize}
\item
  ضعیف
\item
  متوسط
\item
  بالا
\end{itemize}\tabularnewline
سلیقه & \begin{itemize}
\item
  افرادی که مد و فشن توجه می‌کنند
\item
  افرادی که مد برای آنها اهمیت ندارد
\end{itemize}\tabularnewline
\bottomrule
\end{longtable}

جدول 1

با توجه به مطالعات انجام شده بازار هدف خود را با توجه به ویژگی سن افراد
جوان در نظر می‌گیریم. علت این موضوع است که افراد جوان ارتباط بسیار زیادی
با تکنولوژی و اینترنت دارند و به همین جهت این دست از افراد برای یک
فروشگاه آنلاین هدف بسیار مناسب‌تری هستند.

از لحاظ سطح دارمد بررسی‌های ما نشان می‌دهد که افراد با سطح درامدی بالای
بیشتر ترجیح می‌دهند تا کالاهای مدنظر خود را به صورت حضوری و از مکان‌های
خاصی انجام دهند و از سویی دیگر افراد با سطح دارمد ضعیف نیز کمتر جهت خرید
به فروشگاه‌های اینترنتی رجوع می‌کنند به این علت که تصور آنها این است که
در خرید حضوری می‌توانند البسه‌ی مدنظر خود را با قیمت کمتری خریداری کنند.
بنابراین بر اساس این ویژگی افراد با سطح درامد متوسط هدف ما قرار
می‌گیرند.

بر اساس ویژگی سلیقه نیز با توجه به شرایط موجود در بازار و بررسی رقبایی
که انجام داده‌ایم بهتر است تا افرادی که مد و فشن برای آنها مهم است را
مدنظر قرار دهیم به این علت که این افراد برای خرید لباس به روندها توجه
بیشتری دارند و از اینترنت برای یافتن آنها استفاده می‌کنند و از همین طریق
می‌توان آنها را به فروشگاه خود جذب بکنیم.

بنابراین با توجه به توضیحات ذکر شده افراد جوان با سطح دارمد متوسط که مد
و فشن توجه ویژه‌ای دارند بازار هدف فروشگاه آنلاین ما در بخش مشتریان
خواهند بود.

\subsubsection{بخش فروشگاه
(Markets)}\label{ux628ux62eux634-ux641ux631ux648ux634ux6afux627ux647-markets}

در حوزه‌ی پوشاک انواع تولیدکنندگان و فروشگاها از نظر کیفیت کالا، قیمت،
محبوبیت برند در بین مردم، موقعیت جغرافیایی فروشگاه‌ها و ... وجود دارند.

و ما نیز به بررسی برخی از این تفاوت‌ها که اهمیت بیشتری دارند می‌پردازیم.

\begin{longtable}[]{@{}ll@{}}
\toprule
بخش‌های بازار با توجه به ویژگی ذکر شده & ویژگی یا
Attribute\tabularnewline
\midrule
\endhead
\begin{itemize}
\item
  تهران
\item
  شهرستان‌ها
\end{itemize} & جغرافیا و مقصد\tabularnewline
\begin{itemize}
\item
  مناسب
\item
  گران
\item
  ارزان
\end{itemize} & قیمت کالاها\tabularnewline
\begin{itemize}
\item
  متوسط و معقول
\item
  لوکس و بسیار با کیفیت
\item
  ضعیف و بی کیفیت
\end{itemize} & کیفیت کالاها\tabularnewline
\begin{itemize}
\item
  مطابق با مد روز (Fashionable)
\item
  رسمی و اداری
\item
  سنتی و مذهبی
\item
  لباس‌های مشاغل
\item
  دسته‌های متفرقه
\end{itemize} & نوع پوشاک\tabularnewline
\begin{itemize}
\item
  برند در سطح کلان (شهر یا کشور)
\item
  برند در سطح محله
\item
  بدون برند
\end{itemize} & محبوبیت برند\tabularnewline
\begin{itemize}
\item
  خود فروشگاه تولید کننده نیز باشد
\item
  فقط فروشگاه واسط باشد
\end{itemize} & انواع کاربری فروشگاه‌ها\tabularnewline
\bottomrule
\end{longtable}

جدول 2

در حال حاضر به دلیل شرایط محدود در زمینه حمل‌ونقل (پیک موتوری) فقط
می‌توان با فروشگاه‌های در سطح شهر (تهران) وارد مذاکره و عقد قرارداد شد
همچنین پس از تحقیقاتی که بر روی مشتریان و سلایق آنها صورت گرفت باید
فروشگاهایی که دارای کالاهای مطابق با مد روز و قیمت‌های مناسب و کیفیت‌های
معقول هستند پوشش داده شوند. البته اگر این فروشگاه‌ها و اجناسشان در سطح
کلان برند باشند؛ فرایند برندینگ و بازاریابی ما نیز تسهیل می‌شود؛ در
نتیجه هدف فعلی ما این فروشگاه‌ها هستند.

و همچنین به جهت متناسب شدن بیشتر قیمت و مدیریت بهتر موجودی (یعنی حداکثر
توان تامین) و ... ترجیح ما بر این است در حال حاضر فروشگاه‌هایی که خود
تولید کننده‌اند را هدف بازار خود قرار دهیم.

\subsubsection{بخش پیک‌ها
(Couriers)}\label{ux628ux62eux634-ux67eux6ccux6a9ux647ux627-couriers}

در این بخش قصد داریم بر اساس 3 ویژگی جغرافیا، نوع وسیله‌ی نقلیه و نوع
ارسال، به بخش‌بندی پیک‌ها در پلتفرم خود بپردازیم. نتیجه به صورت زیر
می‌باشد:

\begin{longtable}[]{@{}ll@{}}
\toprule
بخش‌های بازار با توجه به ویژگی ذکر شده & ویژگی یا
Attribute\tabularnewline
\midrule
\endhead
شهرستان ها (تهران و سایر شهرستان‌ها) & جغرافیا و مقصد\tabularnewline
\begin{itemize}
\item
  موتور
\item
  ماشین
\item
  ماشین سنگین و وانت
\end{itemize} & وسیله‌ی نقلیه\tabularnewline
\begin{itemize}
\item
  درون شهری
\item
  برون شهری
\end{itemize} & نوع ارسال\tabularnewline
\bottomrule
\end{longtable}

جدول 3

با توجه به بخش‌بندی ارائه‌شده، هدف ما پیک‌های موتوری داخل تهران هستند که
اجناس را به صورت درون شهری جا به جا می‌کنند.

\subsection{حوزه‌های ارزش
ذی‌نفعان}\label{ux62dux648ux632ux647ux647ux627ux6cc-ux627ux631ux632ux634-ux630ux6ccux646ux641ux639ux627ux646}

در این بخش به بررسی حوزه‌های ارزش ذی‌نفعان کسب‌وکار و پلتفرم مورد
ارزیابی خود خواهیم کرد. نکته مهم در این بخش این است که تنها سه دسته
مشتریان اصلی این پلتفرم را به عنوان ذی‌نفع در نظر می‌گیریم و بر اساس این
فرض از بررسی سایر ذی‌نفعان صرف نظر می‌کنیم.

\subsubsection{حوزه مشتری
(Users)}\label{ux62dux648ux632ux647-ux645ux634ux62aux631ux6cc-users}

محدوده‌ی ارزش ذی نفعان برای بخش مشتریان یا همان خریداران به شرح جدول زیر
می‌باشد:

\begin{longtable}[]{@{}ll@{}}
\toprule
توضیح & زمینه\tabularnewline
\midrule
\endhead
دیگر نیاز به هزینه رفت و آمد و صرف وقت برای خرید لباس نیست. & کاهش هزینه
(Cost reducer)\tabularnewline
در بازارهای آنلاین که بازار به شد رقابتی است و همچنین عرضه کنندگان کالا
دیگر هزینه‌هایی از قبیل اجاره مغازه و ... را نمی‌پردازند، قیمت محصولات
ارزان‌تر است. &\tabularnewline
کاهش زمان خرید برای مشتری & توانمندسازی (Enablement)\tabularnewline
ایجاد خرید ایمن و حفظ سلامتی مشتری در شرایط کرونا &\tabularnewline
تنوع بالای محصولات جهت بررسی و خرید &\tabularnewline
از نظر اجتماعی خرید اینترنتی یک خرید سبز (سازگار با محیط زیست) است و
باعث ایجاد تمایز برای این نوع خریداران از نظر اجتماعی می‌شود. &
متمایزکننده (Differentiator)\tabularnewline
\bottomrule
\end{longtable}

جدول 4 -- حوزه‌ی ارزش مشتریان

\subsubsection{حوزه فروشگاه
(Markets)}\label{ux62dux648ux632ux647-ux641ux631ux648ux634ux6afux627ux647-markets}

محدوده‌ی ارزش ذی‌نفعان برای بخش فروشگاه‌ها به شرح جدول زیر می‌باشد:

\begin{longtable}[]{@{}ll@{}}
\toprule
توضیح & زمینه\tabularnewline
\midrule
\endhead
به علت عدم نیاز به تبلیغ و هزینه برای جذب مشتری هزینه‌ها کاهش پیدا
می‌کند. & کاهش هزینه (Cost reducer)\tabularnewline
دیگر نیاز به پرداخت دستمزد برای فروشنده در مغازه نیست و بازار به بازار
آنلاین تبدیل می‌شود. &\tabularnewline
همچنین نیاز به پرداخت اجاره مغازه و ... نیست. &\tabularnewline
گسترش بازار در سطح کل شهر & توانمندسازی (Enablement)\tabularnewline
امکان برگزاری کمپین تبلیغاتی بدون داشتن زیرساخت فنی (یعنی در اصل از طریق
سایت ما می‌توانند کمپین برگزار کنند) &\tabularnewline
در روزگار کرونا که یکی از چالش‌های اساسی مشتریان عدم حضور در بازار است،
وارد شدن به فضای فروش آنلاین می‌تواند یک فروشگاه را از رقبای خودش متمایز
کند. & متمایزکننده (Differentiator)\tabularnewline
\bottomrule
\end{longtable}

جدول 5 -- حوزه‌ی ارزش فروشگاه‌ها

\subsubsection{حوزه پیک‌ها
(Couriers)}\label{ux62dux648ux632ux647-ux67eux6ccux6a9ux647ux627-couriers}

محدوده‌ی ارزش ذی‌نفعان برای بخش پیک‌ها به شرح جدول زیر می‌باشد:

\begin{longtable}[]{@{}ll@{}}
\toprule
توضیح & زمینه\tabularnewline
\midrule
\endhead
از طریق کاهش زمان جست و جو برای مشتری & کاهش هزینه (Cost
reducer)\tabularnewline
مورد هدف قرار دادن بازار جدیدی که ممکن بود در گذشته نسبت به آن آگاهی
نداشته باشند. & توانمندسازی (Enablement)\tabularnewline
امکان همکاری با کارفرماهای مختلف و مقید نشدن به یک کارفرمای خاص
&\tabularnewline
ایجاد تمایز بین پیک و سایر رقبا از طریق استفاده از تکنولوژی‌های به روز &
متمایزکننده (Differentiator)\tabularnewline
\bottomrule
\end{longtable}

جدول 6 -- حوزه ارزش پیک‌ها

\subsection{ویژگی‌های منحصر به
فروش}\label{ux648ux6ccux698ux6afux6ccux647ux627ux6cc-ux645ux646ux62dux635ux631-ux628ux647-ux641ux631ux648ux634}

با توجه به بررسی‌های انجام و شناسایی حوزه‌های ارزش ذی‌نفعان، ویژگی‌های
منحصر به‌فروش کسب‌وکار باید به شرح زیر باشد:

\begin{enumerate}
\def\labelenumi{\arabic{enumi})}
\item
  User friendly بودن اپلیکیشن محصول
\item
  قیمت مناسب محصول برای هر سه دسته‌ی فروشنده، مشتری و پیک
\item
  پشتیبانی از IOS و اندروید و وجود نسخه‌ی وب
\item
  مدیریت ارتباطات مشتریان یا CRM قوی برای فروشنده، مشتری و پیک
\item
  ارائه‌ی نسخه‌های به روز رسانی شده با رفع ایرادات موجود در نسخ قبلی در
  مدت زمان کوتاه
\item
  استفاده از گیمیفیکیشن برای پیک‌ها، مشتریان و فروشندگان (برای مثال
  استفاده از اسکوربورد)
\item
  ارائه‌ی کارت سوخت یا سهیمه‌ی سوخت به پیک‌ها
\item
  امکان مقایسه‌ی محصولات فروشگاه‌های مختلف برای مشتریان
\item
  امکان ردیابی پیک موتوری هنگام تحویل سفارش توسط مشتری
\item
  استفاده از شماره‌های ناشناس برای پیک و مشتری هنگام تحویل سفارش ثبت‌شده
\item
  امکان اطلاع از موجودی کالاها در فروشگاه‌ها توسط مشتری
\item
  وجود سبد خرید برای مشتری و ذخیره کردن کالاهایی که در آینده قصد خرید
  آن‌ها را دارد.
\item
  امکان مرجوع کردن کالا توسط مشتری
\item
  امکان تبلیغ کردن فروشگاه‌ها به فرمتی مانند گوگل ادز
\item
  امکان تحلیل رفتار مشتریان و ارائه‌ی آن به فروشگاه‌ها
\item
  امکان اعلام شعب جدید فروشگاهی که از قبل در پلتفرم حضور دارد
\item
  امکان نمایش تخفیف‌های خاص به درخواست فروشگاه‌ها
\end{enumerate}

\section{داستان کاربری (User
Story)}\label{ux62fux627ux633ux62aux627ux646-ux6a9ux627ux631ux628ux631ux6cc-user-story}

در این بخش نیز همانند بخش‌های پیشین مسئله را از سه دید مشتری، فروشگاه و
پیک مورد ارزیابی قرار می‌دهیم و برای هر دسته از مشتریان داستان‌های
کاربری آنها را ذکر می‌کنیم و بر اساس آنها ویژگی‌های خاص پلتفرم از نگاه
هر دسته را ذکر می‌کنیم.

\subsection{بخش مشتری
(Users)}\label{ux628ux62eux634-ux645ux634ux62aux631ux6cc-users-1}

در این بخش باید با توجه به پرسونای مشتریان خود داستان‌های کاربری متفاوتی
را ذکر کنیم تا بتوانیم بر اساس آنها ویژگی‌های مهم پلتفرم خود را شناسایی
بکنیم. مشتریان ما کاربران جوانی هستند که توقع آنها از یک پلتفرم فروش
لباس این است که تنوع قابل قبولی از کالا و خدمات برای آنها در نظر گرفته
شود. همچنین رابط کاربری مناسبی نیز باید برای آنها طراحی شود. با توجه به
موارد گفته شده داستان‌های کاربری زیر را می‌توان ذکر کرد:

\subsubsection{ویژگی‌های
خاص}\label{ux648ux6ccux698ux6afux6ccux647ux627ux6cc-ux62eux627ux635}

با توجه به داستان‌های کاربری ذکر شده برای بخش مشتریان پلتفرم‌ما به صورت
کلی به نظر می‌آید که باید ویژگی‌های زیر در طراحی ما مورد توجه قرار گیرد:

\begin{itemize}
\item
  دسته‌بندی محصولات و ارائه‌ی یک فهرست مدون از محصولات در صفحه‌ی اول
\item
  ارائه‌ی امکان سبد خرید برای هر کاربر که ساختار آن به صورت پویا باشد
\item
  ارائه‌ی محصولات و لباس‌های متنوع بر اساس فصل‌های مختلف سال
\item
  ایجاد یک صفحه‌ی اختصاصی برای هر محصول و ارائه‌ی اطلاعات دقیق لباس در
  آن صفحه
\end{itemize}

\subsection{بخش فروشگاه
(Markets)}\label{ux628ux62eux634-ux641ux631ux648ux634ux6afux627ux647-markets-1}

در این بخش به بررسی داستان کاربری از دیدگاه فروشگاه‌های لباس فروشی
می‌پردازیم تا در زمینه ویژگی‌هایی که این کاربر نیاز دارد، به جمع بندی
برسیم و برایمان روشن‌تر باشد چه نیازهایی وجود دارد و اولویت نیازها به چه
صورت است:

\subsubsection{ویژگی‌های
خاص}\label{ux648ux6ccux698ux6afux6ccux647ux627ux6cc-ux62eux627ux635-1}

با توجه به داستان‌های کاربری ذکر شده برای فروشگاه‌های لباس فروشی به صورت
کلی به نظر می‌آید که باید ویژگی‌های زیر در طراحی ما مورد توجه قرار گیرد:

\begin{itemize}
\item
  امکان ثبت و یا حذف هر محصول از طریق پنل خود فروشگاه و هم‌چنین امکان
  ویرایش هر قسمت از اطلاعات محصول (قیمت، توضیحات، تنوع‌ها، عکس و ...)
\item
  باید علاوه بر جمع آوری اطلاعات فروش هر پنل برای خودمان، یک بستر را
  برای اتومات کردن امور حسابداری هر فروشگاه فراهم کنیم.
\item
  باید امکان ثبت تخفیف را با یک UI جذاب به جهت جذب مخاطبین برای
  فروشگاه‌ها ایجاد کنیم.
\item
  و همچنین خوب است یک امکان جهت پروموت کردن محصولات برای فروشگاه‌ها
  فراهم شود.
\end{itemize}

\subsection{بخش پیک‌ها
(Couriers)}\label{ux628ux62eux634-ux67eux6ccux6a9ux647ux627-couriers-1}

\textbf{پیک‌های موتوری افرادی هستند که وظیفه‌ی رساندن کالاها از
فروشگاه‌ها به مشتریان را دارند. این افراد از طریق اپلیکیشن پلتفرم
تقاضاهای ورودی را بررسی می‌کنند و در صورت تمایل می‌پذیرند. لازم به ذکر
است که اطلاعات ارسال محصول به آن‌ها نمایش داده می‌شود. سپس توسط اپلیکیشن
به مقصد مورد نظر راهنمایی می‌شوند. دریافت وجه می‌تواند به صورت نقدی و
آنلاین باشد. حالت دوم هنگامی رخ می‌دهد که مشتری قصد ارجاع کالا به
فروشگاه را داشته باشد.}

\subsubsection{ویژگی‌های
خاص}\label{ux648ux6ccux698ux6afux6ccux647ux627ux6cc-ux62eux627ux635-2}

\begin{itemize}
\item
  امکان پرداخت آنلاین مشتریان به حساب پیک موتوری که پیک موتوری بتواند آن
  را از حساب کاربری خود پیگیری کند.
\item
  امکان نمایش قیمت با توجه به مسیر قبل از قبول یا رد سفارش
\item
  نمایش مبدا و مقصد با کمک نقشه‌ (استفاده از API یا اتصال به
  اپلیکیشن‌های مسیریابی مانند بلد)
\item
  امکان دریافت وجه نقد و ثبت دریافت آن در سیستم در حساب کاربری پیک
  موتوری
\end{itemize}

\section{نیازمندی‌های کاربردی
چیست؟}\label{ux646ux6ccux627ux632ux645ux646ux62fux6ccux647ux627ux6cc-ux6a9ux627ux631ux628ux631ux62fux6cc-ux686ux6ccux633ux62a}

در این بخش می‌خواهیم با استفاده از مدل تهیه‌ی سند SRS به معرفی دقیق
نیازمندی‌های کاربردی یا همان Functional Requirementهای کسب‌وکار فرضی خود
بپردازیم. ابتدا باید تعریف بسیار ساده و مناسبی از نیازمندی‌های کاربردی
داشته‌باشیم. منظور از این دسته از نیازمندی‌ها، مواردی هستند که باید در
سیستم و پلتفرم خود در نظر بگیریم به این دلیل که این خصیصه‌ها و ویژگی‌ها
نیازهای اصلی کاربر از یک کسب‌وکار آنلاین فروش لباس می‌باشد و به عبارت
بهتر با این نیاز‌مندی‌ها سیستم می‌تواند عمکلرد خود را داشته باشد و با
استفاده آنها است که رفتار سیستم ما تعریف می‌شود. تصویر زیر به خوبی منظور
ما را از FRها می‌رساند:

\includegraphics[width=6.50000in,height=1.29492in]{media/image3.png}

تصویر 2

حال که تعریف مناسبی از نیازمندی‌های کاربردی یک سیستم و کسب‌وکار داشتیم
با الگو گرفتن از مدل SRS به معرفی آنها برای کسب‌وکار خود می‌پردازیم:

\section{معرفی سیستم و محدوده‌ی
آن}\label{ux645ux639ux631ux641ux6cc-ux633ux6ccux633ux62aux645-ux648-ux645ux62dux62fux648ux62fux647ux6cc-ux622ux646}

در اولین قدم باید تعریف مناسبی از سیستم خود داشته باشیم تا بتوانیم بر
اساس آن پیش برویم. در این سیستم ما پلتفرم یک کسب‌وکار را مورد بررسی قرار
خواهیم داد که در آن افراد می‌توانند با مراجعه به وب‌سایت فروشگاه ما
کالای مدنظر خود را خریداری کنند. بنابراین اولین فرضیه ما در بررسی این
است که تنها کانال ارتباطی و توزیع موجود وب‌سایت کسب‌وکار ما می‌باشد و
اپلیکیشن تلفن همراه برای سامانه در نظر گرفته نشده‌است. این پلتفرم سه
دسته از افراد را به یکدیگر وصل می‌کند و به عبارت بهتر به سه دسته از
مشتریان خدمت می‌رساند. دسته‌اول کسب‌وکار و فروشندگانی لباسی هستند که
باید از طریق عقد قرارداد حضوری مجوز فعالیت در این پلتفرم را دریافت
بکنند. دسته‌ی دوم مشتریان و خریداران عادی هستند که با مراجعه به وب‌سایت
می‌خواهند لباس و کالای مدنظر خود را خریداری کنند. دسته‌ی پایانی مشتریان
پلتفرم نیز پیک‌هایی هستند که وظیفه‌ی آنها انتقال کالاهای خریداری شده به
خریداران است که از طریق سامانه‌های مکان‌یابی سازمان‌دهی می‌شوند.

با توجه به توضیحات ذکر شده هدف و وظیفه‌ی این پلتفرم این است تا از طریق
تسهیل امر فروش فروشندگان لباس خلق ارزش بکند تا کاربران عادی بتوانند به
سادگی و رضایت مناسب خرید خود را انجام دهند. بنابراین مرز سیستم خود را در
سه دسته افرادی که از پلتفرم استفاده می‌کنند محدود می‌کنیم و ارزش
مبادله‌ای نیز خرید و فروش لباس می‌باشد.

\subsection{ویژگی‌ها و خصوصیات
مشتریان}\label{ux648ux6ccux698ux6afux6ccux647ux627-ux648-ux62eux635ux648ux635ux6ccux627ux62a-ux645ux634ux62aux631ux6ccux627ux646}

همانطور که پیش‌تر نیز ذکر کردیم مشتریان ما سه دسته‌ی زیر می‌باشند:

\begin{itemize}
\item
  خریداران و مشتریان عادی
\item
  فروشندگان البسه
\item
  پیک‌های موتوری
\end{itemize}

همانطور که در بخش‌های قبل‌تر این سند نیز تحلیل کردیم هر کدام از این دسته
از افراد ویژگی‌های خاصی دارند که در طراحی و شناسایی نیازمندی‌ها باید به
آن توجه ویژه‌ای داشته‌ایم باشیم که از تکرار مجدد آنها صرف نظر می‌کنیم.
اما نکته‌ی مهم در این بخش این است که تمامی این افراد با تکنولوژی آشنا
هستند و توانایی استفاده از اینترنت به صورت‌ حداقلی را دارا هستند و
می‌توانند از وب‌سایت ما استفاده بکنند و توانایی انجام اقداماتی مانند
ایمیل ساختن را به صورت کامل دارند. بنابراین این فرضیه را به ازای تمامی
کاربران خود باید در نظر بگیریم.

\section{نیازمندی‌های
کاربردی}\label{ux646ux6ccux627ux632ux645ux646ux62fux6ccux647ux627ux6cc-ux6a9ux627ux631ux628ux631ux62fux6cc}

حال در این مرحله به توجه به توضیحاتی که ارائه شد به معرفی نیازمندی‌های
کاربردی می‌پردازیم. هر نیازمندی در سه گام معرفی می‌شود. در گام اول
توضیحاتی مختصری در مورد ویژگی و اهمیت آن ارائه می‌دهیم. در گام بعدی لیست
فعالیت‌هایی که کاربر انجام می‌دهد را ذکر می‌کنیم و در نهایت در گام نهایی
نیازمندی کاربردی را مورد تحلیل و بررسی می‌کنیم.

\begin{enumerate}
\def\labelenumi{\arabic{enumi}.}
\item
  سناریو و ویژگی اول

  \begin{enumerate}
  \def\labelenumii{\arabic{enumii}.}
  \item
    توضیح کوتاه و اهمیت: ثبت‌نام مشتریان و خریداران محصول در پلتفرم و
    تأیید آنها -- اهمیت: بالا
  \item
    لیست مراحل سناریو:

    \begin{enumerate}
    \def\labelenumiii{\arabic{enumiii}.}
    \item
      کاربر وارد وب‌سایت می‎شود و لینک ثبت‌نام را انتخاب می‌کند.
    \item
      سپس وارد محیطی می‌شود تا بتواند اطلاعات خود را وارد بکند.
    \item
      ایمیل یا تلفن همراه خود را وارد می‌کند و منتظر ارسال پیامک یا
      ایمیل تأییدیه می‌ماند.
    \item
      ایمیل تأییدیه توسط کاربر باز می‌شود و مرحله‌ی ثبت‌نام به اتمام
      می‌رسد.
    \item
      اطلاعات کاربر در دیتابیس سیستم ثبت می‌شود.
    \end{enumerate}
  \item
    \textbf{نیازمندی‌های کاربردی:} با توجه به مواردی که ذکر شد اول
    نیازمندی که باید متناسب با این بخش تعریف شود یک لینک ثبت‌نام واضح و
    مناسب در صفحه‌ی اول وب‌سایت می‌باشد. در ادامه‌ی آن نیز باید یک فرم
    مناسب جهت دریافت اطلاعات ثبت‌نام کاربران تعریف شود. در قدم بعدی باید
    سیستم از سامانه‌ای پشتیبانی کند که می‌تواند ایمیل یا پیامک تأییدیه
    ارسال بکند. در پایان نیز از بعد فنی، سیستم باید از یک دیتابیس بسیار
    مناسب بهره ببرد تا بتوان اطلاعات مشتریان را در آن ذخیره کرد.
  \end{enumerate}
\item
  سناریو و ویژگی دوم

  \begin{enumerate}
  \def\labelenumii{\arabic{enumii}.}
  \item
    توضیح کوتاه و اهمیت: ثبت‌نام صاحبان فروشگاه در پلتفرم -- اهمیت: بالا
  \item
    لیست مراحل سناریو:

    \begin{enumerate}
    \def\labelenumiii{\arabic{enumiii}.}
    \item
      صاحب فروشگاه از طریق وب‌سایت اطلاعاتی مانند آدرس کسب‌وکار ما را به
      دست می‌آورد.
    \item
      سپس زمان یک جلسه را رزرو می‌کند تا از این طریق مراجعه حضوری داشته
      باشد.
    \item
      پس از انجام مذاکره عقد قرارداد انجام می‌شود.
    \item
      اطلاعات مهم کاری مانند ساعت کاری و محصولات در سامانه ثبت می‌شود.
    \item
      در نهایت یک حساب کاربری جامع که فروشنده توانایی شخصی‌سازی داشته
      باشد به او تخصیص داده می‌شود تا بتواند محصولات خود را با شرایط
      مدنظر خود به فروش برساند.
    \end{enumerate}
  \item
    \textbf{نیازمندی‌های کاربردی:} با توجه به قدم‌های ذکر شده ابتدا باید
    در وب‌سایت خود یک صفحه‌ی بسیار مناسب طراحی بکنیم که اطلاعات ارتباطی
    و تماسی ما برای فروشندگان در آن به نمایش در بیاید. سپس در صورت امکان
    باید یک فرم یا صفحه‌ی مخصوص جهت رزرو زمان جلسه برای فروشندگان در نظر
    گرفته شود تا برنامه‌ریزی مذاکرات به خوبی پیش برود. برای هر فروشنده
    باید یک فرم و فضای ورود اطلاعات مناسب در نظر گرفته شود تا اپراتورهای
    وب‌سایت ما پس از عقد قرارداد اطلاعات را در آن ثبت کنند، بنابراین به
    یک دیتابیس اطلاعاتی مناسب نیاز خواهیم داشت. آخرین نیازمندی کاربردی
    این بخش نیز مربوط به حساب کاربری فروشندگان است. با توجه به شرایط
    آنها، باید برای هر فروشنده یک صفحه‌ی جداگانه تخصیص داده شود تا
    قابلیت شخصی‌سازی مناسبی را داشته باشد.
  \end{enumerate}
\item
  سناریو و ویژگی سوم

  \begin{enumerate}
  \def\labelenumii{\arabic{enumii}.}
  \item
    توضیح کوتاه و اهمیت: ثبت‌نام پیک‌های موتوری در پلتفرم -- اهمیت:
    متوسط
  \item
    لیست مراحل سناریو:

    \begin{enumerate}
    \def\labelenumiii{\arabic{enumiii}.}
    \item
      پیک موتوری از طریق وب‌سایت اطلاعات تماس را به دست ‌می‌آورد.
    \item
      از طریق رزرو آنلاین زمانی را جهت مراجعه انتخاب می‌کند.
    \item
      پس از مراجعه اطلاعات وسیله‌ی نقلیه او در سامانه ثبت و احراز هویت
      انجام می‌شود.
    \item
      در آخرین گام نیز وسیله‌ی نقلیه‌ی او مجهز به مکان‌یاب می‌شود.
    \end{enumerate}
  \item
    \textbf{نیازمندی‌های کاربردی}: همانند ویژگی پیشین در این بخش نیز
    نیازمند یک صفحه‌ی «تماس با ما» خواهیم داشت که اطلاعات ارتباطی کاملی
    در آن وجود داشته باشد. یک کانال مجزا جهت زمان‌دهی به پیک‌ها جهت
    مراجعه حضوری باید در نظر گرفته شود تا بتوانند زمان مورد نظر خود را
    رزرو کنند. این نیازمندی می‌تواند در قالب یک فرم آنلاین در نظر گرفته
    شود. جهت ثبت‌نام و احراز هویت به دو مورد نیاز خواهیم داشت. در ابتدا
    باید سیستم ما از طریق برخی از APIها به سازمان‌هایی مانند راهنمایی و
    رانندگی، پلیس 10+ و ثبت احوال متصل باشد تا بتوانیم فرایند احراز هویت
    رانندگان پیک‌های موتوری را انجام دهیم. در ادامه نیز باید دیتابیس
    اطلاعاتی مناسبی را جهت ثبت اطلاعات آنها در نظر بگیریم. در پایان نیز
    باید یک اپلیکیشن مناسب جهت اجرای فرایند مکان‌‌یابی و ثبت سفارش‌های
    رانندگان پیک‌ها در نظر بگیریم. بنابراین برای این دسته از مشتریان خود
    یک اپلیکیشن موبایل نیاز خواهیم داشت.
  \end{enumerate}
\item
  سناریو و ویژگی چهارم

  \begin{enumerate}
  \def\labelenumii{\arabic{enumii}.}
  \item
    توضیح کوتاه و اهمیت: عملیات خرید از دید مشتری -- اهمیت: بالا
  \item
    لیست مراحل سناریو:

    \begin{enumerate}
    \def\labelenumiii{\arabic{enumiii}.}
    \item
      کاربران ابتدا بر اساس برخی از ویژگی‌ها مانند فاصله آنها تا محل
      مورد نظر، هزینه‌ی پیک و نوع محصولات فروشگاه‌های مورد نظر خود را
      انتخاب می‌کنند و وارد صفحه‌ی آن می‌شوند.
    \item
      سپس لباس‌ها و اجناس مورد نظر خودشان را وارد سبد خرید می‌کنند.
    \item
      سیستم باید به صورت برخط موجودی کالا را چک کند و در صورت تأیید کالا
      را به سبد خرید اضافه بکند.
    \item
      در ادامه عملیات خرید آنها می‌تواند شروع شود و از طریق درگاه پرداخت
      اینترنتی می‌توانند عملیات خرید را نهایی کنند.
    \item
      اگر خرید با موفقیت انجام شود پیام تأیید نمایش داده می‌شود و خرید
      نهایی می‌شود.
    \item
      در غیر این صورت محصول در سبد خرید باقی می‌ماند تا مشتری دوباره
      تلاش بکند.
    \end{enumerate}
  \item
    \textbf{نیازمندی‌های کاربردی:} اولین و شاید یکی از مهمترین
    نیازمندی‌های کاربردی این پلتفرم این است که مشتریان بتوانند
    فروشگاه‌ها را بر اساس برخی از ویژگی‌های مورد نظرشان فیلتر بکنند و یا
    جست‌وجو کنند. بنابراین یک سامانه‌ی جست‌وجوی هوشمند فروشگاه‌ها بسیار
    حائز اهمیت است. در ادامه همانند تمامی فروشگاه‌های آنلاین باید امکان
    سبد خرید برای مشتریان فراهم شود تا بتوانند کالاهای مدنظر خودشان را
    در آنجا قرار دهند. نکته مهم در این بخش این است که باید سبد خرید به
    موجودی فروشگاه وصل باشد تا بتواند موجود بودن یا عدم موجود بودن کالا
    را بررسی کند. جهت انجام عملیات خرید باید قالبیت انتقال به درگاه خرید
    اینترنتی و بانک فراهم شود بنابراین قابلیت اتصال به درگاه خرید بانک
    دیگر نیازمندی مهم در این بخش است. آخرین مورد نیز هوشمند بودن سبد
    خرید است به این صورت که در صورت شکست عملیات پرداخت سبد خرید دست
    نخورده باقی بماند و تغییری نکند. به این دلیل که در صورت تغییر موجود
    فروشگاه دچار خطا خواهد شد.
  \end{enumerate}
\item
  سناریو و ویژگی پنجم

  \begin{enumerate}
  \def\labelenumii{\arabic{enumii}.}
  \item
    توضیح کوتاه و اهمیت: اجرای فرایند تحویل سفارش -- اهمیت: بالا
  \item
    لیست مراحل سناریو:

    \begin{enumerate}
    \def\labelenumiii{\arabic{enumiii}.}
    \item
      اطلاعات سفارش ثبت شده به فروشنده ارسال می‌شود تا بتواند در
      سریع‌ترین زمان سفارش را آماده بکند.
    \item
      به طور همزمان و با تأیید فروشنده، نزدیک‌ترین پیک باید به طور
      هوشمند انتخاب شود.
    \item
      اطلاعات مربوط به سفارش و آدرس‌های مبداء و مقصد باید برای پیک ارسال
      شود.
    \item
      در صورت عدم تأیید پیک باید فرد دیگری انتخاب شود.
    \item
      در نهایت تحویل سفارش انجام می‌شود.
    \item
      ثبت نظر برای پیک و محصول خریداری‌شده
    \end{enumerate}
  \item
    \textbf{نیازمندی‌های کاربردی}: سامانه ثبت سفارش باید به صورت هوشمند
    طراحی شود تا به صورت خودکار اعلان سفارش را برای فروشگاه ارسال بکند.
    همچنین باید به صورت خودکار فرایند تخصیص پیک برای هر سفارش انجام شود.
    این دو نیازمندی کاربردی باید به‌گونه‌ای طراحی شود تا دخالت انسانی
    وجود نداشته باشد. نیازمندی کاربردی دیگر امکان ثبت نظر برای محصول و
    پیک است که باید به وسیله‌ی مشتری انجام شود. این اطلاعات باید در یک
    دیتابیس اطلاعاتی مناسب ذخیره شود.
  \end{enumerate}
\item
  سناریو و ویژگی ششم

  \begin{enumerate}
  \def\labelenumii{\arabic{enumii}.}
  \item
    توضیح کوتاه و اهمیت: فرایند مرجوع کردن کالا: اهمیت: متوسط
  \item
    لیست مراحل سناریو:

    \begin{enumerate}
    \def\labelenumiii{\arabic{enumiii}.}
    \item
      کاربر در صورت نیاز به مرجوع کردن کالا با پشتیبانی تماس می‌گیرد.
    \item
      بررسی درخواست مشتری به وسیله‌ی تیم پشتیبانی
    \item
      تخصیص پیک و مرجوع کردن کالا به فروشنده
    \end{enumerate}
  \item
    \textbf{نیازمندی‌های کاربردی:} در مرحله تقریباً نیازمندی کاربردی
    جدید وجود ندارد. همانند مواردی که پیش‌تر ذکر کردیم باید یک صفحه‌ی
    ارتباطی مناسب در وب‌سایت طراحی شود تا کاربران به آن دسترسی داشته
    باشند. همچنین جهت مرجوع کردن محصولات نیاز به یک سیستم هوشمند
    حمل‌ونقل داریم تا از آن استفاده بکنیم. سیستم مالی پلتفرم ما نیز باید
    به گونه‌ای طراحی شود تا بتوانیم از طریق آن عودت وجه مالی را انجام
    دهیم.
  \end{enumerate}
\end{enumerate}

\section{محدودیت‌های
محصول}\label{ux645ux62dux62fux648ux62fux6ccux62aux647ux627ux6cc-ux645ux62dux635ux648ux644}

\includegraphics[width=3.35764in,height=3.10486in]{media/image4.png}محدودیت‌های
سه‌گانه‌ای که در هر پروژه ما با آن‌ها روبرو هستیم، زمان، بودجه و دامنه
یا محدوده پروژه است.

ما با بررسی این سه محدودیت که از تئوری محدودیت‌ها (TOC) برگرفته شده‌اند،
می‌توانیم دید صحیح‌تری به پروژه پیدا کنیم و بتوانیم کیفیت بهینه را در
پروژه خود به ارمغان آوریم.

حال به بررسی این موارد در محصول پروژه خود می‌پردازیم.

\textbf{محدودیت اول: بودجه}

بودجه اولین و مهم‌ترین محدودیت تمام پروژه‌های کوچک و بزرگ است. و مدیریت
این محدودیت از اهمیت بالایی برخوردار است به طوری که می‌تواند روی دیگر
محدودیت‌ها تاثیر مستقیم بگذارد.

از محدودیت‌هایی که در پروژه فوق در زمینه بودجه وجود دارند محدودیت منابع
مالی برای جذب نیروی متخصص و بالا بردن تعداد توسعه دهنده‌های وب،
تولیدکنندگان محتوا و ...، محدودیت منابع مالی به جهت مدیریت امور روزانه
مانند بحث مدیریت لجستیک، محدودیت منابع در امور مارکتینگ یعنی هزینه
تبلیغات، برگزاری کمپین‌ها و ... ، و همچنین محدودیت در زمینه امنیت سایت و
... می‌باشد. برای مثال به دلیل مسائل مالی نمی‌توانیم از یک ERP پر هزینه
استفاده کنیم.

\textbf{محدودیت دوم : زمان}

زمان برای هیچ‌کس صبر نمی‌کند! محدودیت زمان، در بازار رقابتی امروز به‌
حدی است که مدیریت‌ موثر زمان، به‌عنوان ابزاری برای نجات سازمان‌ها مطرح
‌است. تفاوت بین پیشرو بودن و تقلیدکردن تنها در زمان اولین تحویل به بازار
است!

به ‌خصوص در پروژه‌ای مانند پروژه‌ی تعریف شده، که سالانه تیم‌های بسیاری
وارد چنین بازارهایی می‌شوند و هرچه این محصول دیرتر آماده شود، کار تیم
برای ورود به بازار سخت‌تر می‌شود.

در چنین مواردی زمان به عنوان یک محدودیت مهم ظاهر می‌شود؛ که در صورت
نادیده گرفته شدن در هر قسمت از کار می‌تواند ما را با ریسک‌های بالایی
روبرو کند.

ما با محدودیت‌های مختلفی در زمان روبرو هستیم؛ محدودیت در زمان ساخته شدن
محصول و بالا آمدن سایت، محدودیت در زمان تبلیغات و جذب مشتریان، یعنی باید
با توجه به سرمایه گذاری که انجام شده در مدتی تعداد مشخصی مشتری را برای
سایت جذب کنیم و تا بازگشت سرمایه طبق برنامه‌ریزی پیش برویم تا متضرر
نشویم، محدودیت در زمان تامین کالا و ارسال سفارش که رعایت نکردن این
محدودیت می‌تواند باعث تجربه بد مشتری شود ، محدودیت در زمان پشتیانی
آنلاین در سایت و ...

\textbf{محدودیت سوم : ابعاد و محدوده پروژه}

این محدودیت می‌تواند جنبه‌های مختلفی داشته باشد و هر کدام از این جنبه‌ها
کار را برای مدیریت پروژه و ارائه محصول با کیفیت سخت می‌کند.

از محدودیت‌هایی که در این بخش می‌توان به آن اشاره کرد محدودیت منابع
انسانی (نفرساعت) یعنی تعداد برنامه‌نویس‌های تیم به جهت بالا آوردن سایت،
محدودیت فروشگاه‌ها و تنوع کالاها، محدودیت لجستیک (ظرفیت کم انبار، محدود
بودن تعداد پیک‌ها)، محدودیت زیرساخت‌ها و تکنولوژی‌هایی که از آن‌ها در
جهت ایجاد و تولید سایت بهره می‌بریم، محدودیت محیط مانند قوانین و مقررات
هر کشور در زمینه تولید سایت‌های فروشگاهی، محدودیت‌های حقوقی، شرایط تحریم
و ... ، محدودیت ابزار و تجهیزات مثل سرورهای سایت و .... است.

نکته‌ای که در متن پروژه به آن اشاره نشده بود ولی مهندسین صنایع موظف‌اند
پس از بیان محدودیت‌ها انجام بدهند این است که باید برای هر یک از این
محدودیت‌ها را به صورت عددی تخمین بزنند و سپس با توجه به اعداد تخمین زده
شده، برنامه‌ی خود را در زمینه اجرا و مدیریت منابع پروژه ارائه دهند.

در ادامه به بررسی non-functional requirementها یا NFR می‌پردازیم که
نشان‌دهنده‌ی محدودیت‌ها در سطح سیستم هستند. در واقع هر نیاز که چگونگی
عملکرد سیستم در هر عملیات خاص را نشان بدهد یک NFR است. اگر این
محدودیت‌ها براورده نشوند سیستم نیازهای کاربران را به درستی برطرف نخواهد
کرد. هر کدام از این محدودیت‌ها در دسته‌بندی‌های زیر قرار می‌گیرند:

\begin{enumerate}
\def\labelenumi{\arabic{enumi}.}
\item
  \includegraphics[width=1.52083in,height=5.12708in]{media/image5.png}مقیاس‌پذیری
\item
  قابلیت اطمینان
\item
  دسترسی
\item
  تنظیمی
\item
  قابلیت بازیابی
\item
  ظرفیت
\item
  نگهداری
\item
  خدمت‌دهی
\item
  امنیت
\item
  مدیریت
\item
  محیطی
\item
  تجمیع داده
\item
  قابلیت همکاری
\item
  قابلیت استفاده
\end{enumerate}

تعدادی از محدودیت‌های ما به صورت زیر است:

\begin{enumerate}
\def\labelenumi{\arabic{enumi})}
\item
  کاربران باید رمز اولیه‌ای که به آن‌ها تخصیص داده شده است را به سرعت
  بعد از ورود موفق عوض کنند و رمز اولیه نباید دوباره قابل استفاده باشد.
\item
  هر تلاش ناکام کاربران برای دسترسی به داده‌ها باید به واحد بازرسی گزارش
  شود.
\item
  وب سایت باید ظرفیت پشتیبانی 20 میلیون کاربر را بدون این که اختلالی در
  آن ایجاد شود داشته باشد.
\item
  نرم‌افزار باید portable یا قابل حمل باشد. به عبارتی تغییر از یک OS به
  OS دیگر برای آن ایرادی ایجاد نکند.
\item
  امنیت اطلاعات و حقوق مالکیت معنوی باید حسابرسی و بازرسی شود.
\item
  رنگ‌های استفاده شده برای مشتریان، پیک‌ها و فروشگاه‌ها متفاوت باشد.
\item
  اندازه‌ی buttonهای استفاده شده مناسب باشد. نه چندان بزرگ و نه کوچک.
\item
  اپیلیکیشن و وب سایت به صورت 24/7/365 قابل استفاده باشد.
\item
  با گسترش کاربران فضای ابری بیشتری احتیاج است. این فضا باید قابل
  براورده کردن باشد.
\item
  امنیت دیتابیس باید نیازمندی‌های HIPPA را برآورد کند.
\item
  طراحی باید به گونه‌ای باشد که کاربران با حداکثر 3 کلیک به اطلاعات
  پروفایل خود دسترسی پیدا کنند.
\item
  تمامی صفحات وب باید در کمتر از 4 ثانیه لود شوند.
\item
  فضای سرور باید قابلیت جای دادن سخت‌افزارها را تا ظرفیت 2 برابر در
  آینده داشته باشد.
\item
  برای کارمندان مشخص و تایید شده دسترسی 24 ساعته به فضای سرور باید وجود
  داشته باشد.
\item
  بک گراند کلیه‌ی صفحات باید \#fff4b6 باشد.
\item
  برای برنامه‌نویسی سیستم نباید از کدهای منسوخ استفاده شود.
\item
  ویرایش اطلاعات توسط کاربران در مدت 2 ثانیه به روزرسانی شود.
\item
  حداکثر تاخیر ارسال ایمیل به کاربران 12 ساعت باشد.
\item
  هر درخواست در کمتر ار 10 ثانیه پردازش شود.
\end{enumerate}

\section{نمودار BurnDown
Chart}\label{ux646ux645ux648ux62fux627ux631-burndown-chart}

در پایان این فاز نیز نمودار burndown اجرای پروژه را نهایی می‌کنیم که
نسخه‌ی نهایی آن به شرح زیر می‌باشد:

Taskboard نیز در پایان این فاز پروژه پیش از اتمام نگراش گزارش نهایی فاز
اول به شرح زیر است:

\includegraphics[width=6.50000in,height=3.09375in]{media/image6.png}

تصویر 3

\subsection{منابع و
مآخذ}\label{ux645ux646ux627ux628ux639-ux648-ux645ux622ux62eux630}

\begin{itemize}
\item
  کتاب MIS
\end{itemize}

\begin{itemize}
\item
  \url{https://www.justinmind.com/blog/user-story-examples/}
\item
  \url{https://www.aha.io/roadmapping/guide/requirements-management/what-is-a-good-feature-or-user-story-template}
\item
  \url{https://learning.oreilly.com/library/view/essential-scrum-a/9780321700407/ch05.html}
\item
  \url{https://learning.oreilly.com/library/view/essential-scrum-a/9780321700407/ch18.html\#ch18}
\item
  \url{https://reqtest.com/requirements-blog/functional-vs-non-functional-requirements/}
\item
  \url{https://qracorp.com/functional-vs-non-functional-requirements/}
\item
  \url{https://www.altexsoft.com/blog/non-functional-requirements/}
\item
  \url{https://www.guru99.com/non-functional-requirement-type-example.html}
\item
  \url{https://reqtest.com/requirements-blog/functional-vs-non-functional-requirements/}
\end{itemize}

\end{document}
